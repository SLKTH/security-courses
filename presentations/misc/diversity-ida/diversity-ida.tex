\documentclass[Screen16to9,17pt]{foils}
\usepackage{zencurity-slides}

\selectlanguage{danish}

% Diversitet er et fælles anliggende for alle i tech v. Henrik Kramselund, Internet-, netværks- og sikkerhedskonsulent

% Radia, Dorothy, Elizabeth og Laura er fornavnene på nogle af internet-, netværks- og sikkerhedskonsulentens rollemodeller inden for IT. Lær mere om dem og vær med, når Kramselund giver et call to action og deler erfaringer med at skabe inkluderende miljøer i tech



\begin{document}

\mytitlepage
{Diversitet er et fælles anliggende for alle i tech}
{Diversity in Tech \& Science IDA}

\link{https://diversity-in-tech-science.ida.dk/}

\hlkprofil

\slide{Men jeg er også en rollemodel}

\begin{quote}
{\bf \LARGE Call to Action}

We need to be aware of the impact of our behaviour on others.

When a person discovers they have made a mistake, they should acknowledge this and apologise.
\end{quote}
Kilde:
\link{https://www.ripe.net/publications/docs/ripe-766}

\begin{list1}
\item Jeg er en rollemodel. Hvad enten jeg vil eller ej
\item Mennesker er sociale væsener -- efterligner hinanden
\vskip 2cm

\item Jeg er ekstremt priviligeret -- og det betyder ansvar!

\item Ændringer kommer ikke af sig selv, vi skal fokusere og gøre
\end{list1}



\slide{Hvad vil vi ha' -- læring, sameksistens, friheden til at være}

\hlkimage{13cm}{IPv6-Hackathon-image-itu-ripe.png}


{\color{red}\faHeart} Billede fra RIPE NCC IPv6 hackathon på ITU November 2017: {\color{red}\faHeart}\\{\footnotesize
\link{https://labs.ripe.net/author/becha/results-of-the-ripe-ncc-hackathon-version-6/}}

\slide{Mine rollemodeller}
\hlkimage{11cm}{pawel-janiak-dxFi8Ea670E-unsplash.jpg}

\begin{list1}
\item Når snakken falder på rollemodeller, vil jeg gerne fokusere på nogle
rollemodeller jeg har

\item Det er mennesker som jeg har lært fra igennem mange år og ser op til

\item Derfor kommer nu en kort hyldest og lister over \emph{Rollemodeller indenfor netværk og IT-sikkerhed}
\end{list1}


\slide{Radia Joy Perlman (born December 18, 1951)}

Kilde:
\link{https://en.wikipedia.org/wiki/Radia_Perlman}\\
\hlkrightimage{5cm}{images/440px-Radia_Perlman_2009.jpg}

\begin{quote}
\vskip -2cm
She is most famous for her invention of the spanning-tree protocol (STP), which is fundamental to the operation of network bridges, while working for DEC. She also made large contributions to many other areas of network design and standardization, such as link-state routing protocols.

Perlman was elected a member of the National Academy of Engineering in 2015 for contributions to Internet routing and bridging protocols.[1]
...

She holds more than 100 issued patents.[13] ... Perlman is the recipient of awards such as Lifetime Achievement awards from Usenix and the Association for Computing Machinery’s Special Interest Group on Data Communication (SIGCOMM).[14]
\end{quote}

Vores Internet bygger på hendes viden, hun er helt central for Internet



\slide{Dorothy Elizabeth Denning, born August 12, 1945}

Kilde: \link{https://en.wikipedia.org/wiki/Dorothy_E._Denning}\\
\hlkrightimage{5cm}{images/Dorothy-Denning-Feb2013-head.jpg}

\begin{quote}
\vskip -2cm
Dorothy Elizabeth Denning, born August 12, 1945, is a US-American information security researcher known for lattice-based access control (LBAC), intrusion detection systems (IDS), and other cyber security innovations. She published four books and over 200 articles. Inducted into the National Cyber Security Hall of Fame in 2012, she is now Emeritus Distinguished Professor of Defense Analysis, Naval Postgraduate School.
...
Detecting intruders is key to protecting computer systems. While at SRI International Dorothy Denning and Peter G. Neumann developed an intrusion detection system (IDS) model using statistics for anomaly detection that is still the basis for intrusion detection systems today.
\end{quote}


Central indenfor netværkssikkerhed, herunder Intrusion Detection begrebet som vi dagligt bruger implementeret i diverse programmer.


\slide{Shon Harris, Mange andre rollemodeller}

.
\hlkrightimage{5cm}{images/Shon-Harris.jpg}

\begin{list2}
\item Shon Harris -- forfatter på DEN CISSP sikkerhedsbog alle brugte, og bruger:\\
\emph{At the time of her death, over 1,000,000 copies of her books had been
sold}\\
\link{https://en.wikipedia.org/wiki/Shon_Harris}
\item Elizabeth D. Zwicky - firewallbogen \emph{Building Internet Firewalls}, O'Reilly
\item Laura Chappell - Wireshark University siden 2006, mange bøger
\item Tina Bird - som jeg lærte logging af på kursus for mange år siden\\
\link{https://www.precision-guesswork.com/tbird-bio.html}
\item {\bf Twitter InfoSec} plus alle de "nye" som jeg følger, lytter til og køber bøger af nu, fra twitter
osv. Violet Blue @violetblue, Amanda Berlin @InfoSystir, VM (Vicky) Brasseur
@vmbrasseur ...
\end{list2}

Internet har givet os adgang til rollemodeller over et bredt spektrum, hvilket er godt.

\slide{Hvordan skaber vi gode miljøer}

\hlkimage{10cm}{bornhack-2022-logo-l.png}

\begin{list2}
\item BornHack, har skabt et virkeligt godt miljø som nydes af alle fra børn/unge til gamle, og alle mennesker
\item Da vi startede i 2016 vidste vi at vi ville lave et miljø -- hvor alle skulle være velkomne.
\item Vi valgte at genbruge store dele af Code of Conduct fra EMF, og lavede BornHack Code of Conduct\\
\link{https://bornhack.dk/conduct/}
\item Det har været en stor success og I er allesammen velkomne til sommer, Aug 3. til 10. 2022
\end{list2}



\slide{RIPE NCC Code of Conduct }

RIPE Code of Conduct
Publication date: 05 Oct 2021

\begin{quote}
Rationale
Our goals in having this Code of Conduct are:
\begin{list2}
\item {\bf To help everyone feel safe and included.} Many people will be new to our community. Some may have had negative experiences in other communities. We want to set a clear expectation that harassment and related behaviours are not tolerated here. If people do have an unpleasant experience, they will know that this is neither the norm nor acceptable to us as a community.

\item {\bf To make everyone aware of expected behaviour.} We are a diverse community; a CoC sets clear expectations in terms of how people should behave.
\end{list2}
\end{quote}
Kilde: {\small
\link{https://www.ripe.net/publications/docs/ripe-766}}

Se også mentorprogrammet fra RIPE NCC\\{\small
\link{https://www.ripe.net/participate/meetings/ripe-meetings/mentoring-programme}}


\slide{"Pac-Man" rule}

\begin{quote}
\vskip -1cm The rule is:\\
{\bf When standing as a group of people, always leave room for 1 person to join your group.} \hskip 3cm More memorably, stand like Pac-Man!
\end{quote}

\hlkimage{6cm}{images/pacman-rule.png}

\begin{quote}
The new person, who has been given permission to join your group, will gather up the courage, and join you! Another important point, the group should now readjust to leave another space for a new person.\\
\link{https://ericholscher.com/blog/2017/aug/2/pacman-rule-conferences/}
\end{quote}

\slide{Erkend, undskyld og iværksat ændringer }

\begin{quote}
{\bf \LARGE Call to Action}

We need to be aware of the impact of our behaviour on others.

When a person discovers they have made a mistake, they should acknowledge this and apologise.
\end{quote}
Kilde:
\link{https://www.ripe.net/publications/docs/ripe-766}

Min anbefaling er at sætte fokus på krænkelser
\begin{list2}
\item Nultolerance overfor krænkende adfærd
\item Tydeliggør støtte og klageinstanser
\item Undervis i håndtering af krænkende adfærd
\end{list2}

Tak.


\end{document}
