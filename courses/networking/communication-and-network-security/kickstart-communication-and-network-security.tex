\documentclass[a4paper,11pt,notitlepage]{report}
% Henrik Lund Kramshoej, February 2001
% hlk@security6.net,
% My standard packages
\usepackage{zencurity-network-exercises}

\begin{document}

\rm
\selectlanguage{english}

\newcommand{\subject}[1]{Communication and Network Security}

%\mytitle{SIEM and Log Analysis}{exercises}
{\LARGE Kickstart: Communication and Network Security}{}


\normal

This material is prepared for use in \emph{\subject} and was prepared by
Henrik Kramselund Jereminsen, \link{http://www.zencurity.com} .
It contains the very basic information to get started!

These course and exercises are expected to be performed in a training setting with network connected systems. The exercises use a number of tools which can be copied and reused after training. A lot is described about setting up your workstation in the Github repositories.

{\bf The main site for materials are: \link{https://github.com/kramse/}}

I try to gather all information there!

So to get kickstarted in this course:
\begin{list2}
\item[\faSquareO] Make sure you can login to Fronter\\
\link{https://kea-fronter.itslearning.com/}\\
Electronic version of this document will be uploaded here!
\item[\faSquareO] Bookmark the main Github page: \link{https://github.com/kramse/}\\
Note: there are two pinned repositories \verb+security-courses+ and \verb+kramse-labs+
\item[\faSquareO] Lecture plan for this course\\
\link{https://zencurity.gitbook.io/kea-it-sikkerhed/net-og-komm-sikkerhed/lektionsplan}\\
(Source is also in Git https://github.com/kramse/kea-it-sikkerhed )
\item[\faSquareO] Slides and exercises booklet -- clone or download single files\\
\link{https://github.com/kramse/security-courses/tree/master/courses/networking/communication-and-network-security}
\item[\faSquareO] Read about setup of exercise systems here\\
\link{https://github.com/kramse/kramse-labs}
\item[\faSquareO] Check BIOS settings - make sure CPU settings have virtualisation turned ON
\item[\faSquareO] Select and install virtualisation software
\item[\faSquareO] Get the books! Either on paper or PDF
\end{list2}

I hope we will have a fun and enjoyable time in this course.

\end{document}
