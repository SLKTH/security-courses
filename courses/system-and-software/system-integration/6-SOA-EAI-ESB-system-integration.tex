\documentclass[Screen16to9,17pt]{foils}
\usepackage{zencurity-slides}
\externaldocument{system-integration-exercises}
\selectlanguage{english}

% Systemintegration

\begin{document}

\mytitlepage
{6. SOA, EAI, ESB - connecting the dots }
{KEA System Integration}


\slide{Plan for today}

\begin{list2}
\item Service-Oriented Architecture (SOA)
\item Read chapters 1-5 in the SOA book, less pages than it seems - large figures on many pages!
\item Exercises in database and cloud computing
\item {\bf Look at hand in assignment!}
\end{list2}

Exercises
\begin{list2}
\item Run PostgreSQL
\item Why go to SOA
\item Cloud Computing Introduction, Cloud Deployment
\item Download the Microservices ebook
\end{list2}


\slide{Todays Agenda - approximate time plan}

\begin{list2}
\item 08:15 - 09:00 Book chapters
\item
\item 09:00 Break
\item 09:15 - 10:00 45m
\item 10:00 Break
\item 10:15 - 11:30
\item 11:30 Lunch Break
\item 12:15 - 13:45
\item Exercises
\item Investigate real-life examples
\end{list2}



\slide{Goals for this week in system integration}

\hlkimage{6cm}{thomas-galler-hZ3uF1-z2Qc-unsplash.jpg}

This weeks goals:
\begin{list2}
\item Get an understanding of the SOA book and SOA
\item Find time to do some exercises, communicate with friends, students and instructor
\end{list2}

Photo by Thomas Galler on Unsplash






\slide{Reading Summary}

SOA ch 1-5:
\begin{list2}
\item CHAPTER 1: Introduction
\item CHAPTER 2: Case Study Backgrounds
\item CHAPTER 3: Understanding Service-Orientation
\item CHAPTER 4: Understanding SOA
\item CHAPTER 5: Understanding Layers with Services and Microservices
\end{list2}

\slide{Service-oriented architecture (SOA)}

\begin{quote}
Service-oriented architecture (SOA) is a style of software design where services are provided to the other components by application components, through a communication protocol over a network. A SOA service is a discrete unit of functionality that can be accessed remotely and acted upon and updated independently, such as retrieving a credit card statement online. {\bf SOA is also intended to be independent of vendors, products and technologies.[1]}

A service has four properties according to one of many definitions of SOA:[2]
\begin{list2}
\item It logically represents a business activity with a specified outcome.
\item It is self-contained.
\item It is a black box for its consumers, meaning the consumer does not have to be aware of the service's inner workings.
\item It may consist of other underlying services.[3]
\end{list2}
\end{quote}
Source:{\footnotesize\\
\url{https://en.wikipedia.org/wiki/Service-oriented_architecture}}



\slide{The SOA Manifesto}

Through our work we have come to prioritize:
\begin{list2}
\item {\bf Business value} over technical strategy
\item {\bf Strategic goals} over project-specific benefits
\item {\bf Intrinsic interoperability} over custom integration
\item {\bf Shared services} over specific-purpose implementations
\item {\bf Flexibility} over optimization
\item {\bf Evolutionary refinement} over pursuit of initial perfection
\end{list2}

Book references, in Appendix D the \emph{SOA Manifesto}\\
\url{www.soa-manifesto.org.}

I recommend reading the explanation in \emph{The SOA Manifesto Explored}

% SOA book chapters 1-5


\slide{Book: Service-Oriented Architecture}

\hlkimage{5cm}{thomas-erl-book.png}
\emph{Service‑Oriented Architecture: Analysis and Design for Services and Microservices}, Thomas Erl, 2017
ISBN: 978-0-13-385858-7

If you go to informit.com,
register and log in or create an account, you can get a substantial discount on the eBook in PDF format. Enter the product ISBN, 9780133858587 and I filled out a small survey.


\slide{Chapter 1: Introduction}

\begin{list2}
\item Since publishing the first edition, they published a series of 11 books of which 8 were dedicated to SOA
\item Multiple parts updated from that work, so editions are different!
\item Patterns have been identified and presented on \url{www.soapatterns.org} which now redirect\\
 to \url{https://patterns.arcitura.com/soa-patterns} as part of
a SOA Certified Professional (SOACP) program
\item Similarly for cloud computing \url{www.cloudpatterns.org} and Big
Data \url{www.bigdatapatterns.org}
\item Note: getting certified is not a requirement for working with SOA
\end{list2}


\slide{Capitalization for Principles, Constraints, and Patterns}


\begin{quote}
To maintain an immediately recognizable distinction between constraints, principles, and patterns throughout this book, each uses a different delimiter for page numbers. The page number for each constraint is displayed in curly braces, for each principle it is placed in rounded parentheses, and for patterns, square brackets are used, as follows:

\begin{list2}
\item Principle Name (page number)
\item Constraint Name \{page number\}
\item Pattern Name [page number]
\end{list2}
For example, the following statement first references a service-orientation design principle, then an SOA design pattern, and finally a REST constraint:
“...the Service Loose Coupling (293) principle is supported via the application of the Decoupled
Contract [337] pattern and the Stateless \{308\} constraint ...”
\end{quote}
Source: \emph{Service‑Oriented Architecture: Analysis and Design for Services and Microservices}, Thomas Erl, 2017

\slide{SOA in Denmark}

\begin{list2}
\item SOA is used in Denmark too
\item Misc links:
\item HVAD ER SOA? \url{http://arkitekturguiden.digitaliser.dk/soa/hvad-er-soa}
\item  Danish National Interoperability Framework (NIF)\\
\url{https://joinup.ec.europa.eu/sites/default/files/inline-files/NIFO%20-%20Factsheet%20Denmark_12_2015.pdf}
" Denmark describes an architecture
 based on ServiceOriented architecture principles and puts forward standards for Service-Oriented infrastructure."

\item Jobs relating to SOA in Denmark too
\end{list2}



\slide{Chapter  2: Case Study Backgrounds}

\begin{list2}
\item Very short chapter with 2 back stories for two different organizations
\item You need to make one for your own assignment:\\
\emph{Hand-in assignment I: Describe the system environment for an organisation}
\end{list2}



\slide{Chapter 3: Understanding Service-Orientation}

\begin{quote}
{\bf Chapter 3: Understanding Service-Orientation}\\
This chapter provides detailed coverage of the service-orientation design paradigm, including its underlying design philosophy and design principles, as well as a comparison to traditional silo-based design approaches. The chapter concludes with coverage of typical critical success factors for adopting service-orientation within organizations.
\end{quote}
Source: \emph{Service‑Oriented Architecture: Analysis and Design for Services and Microservices}, Thomas Erl, 2017



\slide{Services in Business Automation}

\hlkimage{12cm}{soabook-3.2-capabilities.png}

\begin{list2}
\item Different capabilities are composed/combined into a business/service
\item with computers we use the terms \emph{service} and \emph{service contract} which includes published Application programming interface (API)
\end{list2}



\slide{Common technologies}

We already mentioned some of the technologies used for this:
\begin{list2}
\item Extensible Markup Language (XML) used in Web service (WS) with Web Services Description Language (WSDL) and Simple Object Access Protocol (SOAP)
\item Allowing us to do Remote Method Invocation (RMI) aka Remote Procedure Call (RPC)
\item Sometimes incorporating XML schema (XSD), Extensible Stylesheet Language Transformations (XSLT) and even producing HyperText Markup Language (HTML) documents or perhaps JavaScript Object Notation (JSON)
\item Too many acronyms? Use Wikipedia!
\item See the patterns and recognize common use-cases
\item Or let Camel and Python convert data \smiley
\end{list2}

\slide{Services Are Collections of Capabilities}

\hlkimage{16cm}{soabook-3-5-service-collection-capabilities.png}
Source: \emph{Service‑Oriented Architecture: Analysis and Design for Services and Microservices}, Thomas Erl, 2017



\slide{Service-Orientation is a Design Paradigm}

\begin{quote}
A design paradigm is an approach to designing solution logic. When building distributed solution logic, design approaches revolve around a software engineering theory known as the “separation of concerns.” In a nutshell, this theory states that a larger problem is more effectively solved when decomposed into a set of smaller problems or concerns. This gives us the option of partitioning solution logic into capabilities, each designed to solve an individual concern. Related capabilities can be grouped into units of solution logic.
\end{quote}

\begin{list2}
\item A \emph{service composition} is a coordinated aggregate of services
\item A \emph{service inventory} is an independently standardized and governed collection of complementary services within a boundary that represents an enterprise or a meaningful segment of an enterprise.
\end{list2}

Source: \emph{Service‑Oriented Architecture: Analysis and Design for Services and Microservices}, Thomas Erl, 2017


\slide{Problems Solved by Service-Orientation}

\hlkimage{16cm}{soabook-3-14-excess-logic.png}

\begin{list2}
\item Redundant functionality is costly in the long run
\item Integration Becomes a Constant Challenge
\end{list2}

Source: \emph{Service‑Oriented Architecture: Analysis and Design for Services and Microservices}, Thomas Erl, 2017

\slide{The Need for Service-Orientation}

The consistent application of the eight design principles we listed earlier results in the widespread proliferation of the corresponding design characteristics:

\begin{list2}
\item increased consistency in how functionality and data is represented
\item reduced dependencies between units of solution logic
\item reduced awareness of underlying solution logic design and implementation details
\item increased opportunities to use a piece of solution logic for multiple purposes
\item increased opportunities to combine units of solution logic into different
confi gurations
\item increased behavioral predictability
\item  increased availability and scalability
\item increased awareness of available solution logic
\end{list2}

Source: \emph{Service‑Oriented Architecture: Analysis and Design for Services and Microservices}, Thomas Erl, 2017


\slide{Reusable Solution Logic}

\hlkimage{15cm}{soabook-3-17-reusable.png}

Source: \emph{Service‑Oriented Architecture: Analysis and Design for Services and Microservices}, Thomas Erl, 2017




\slide{3.5 Four Pillars of Service-Orientation}

The four pillars of service-orientation are
\begin{list2}
\item Teamwork – Cross-project teams and cooperation are required.
\item Education – Team members must communicate and cooperate based on common knowledge and understanding.
\item Discipline – Team members must apply their common knowledge consistently.
\item Balanced Scope – The extent to which the required levels of Teamwork, Education, and Discipline need to be realized is represented by a meaningful yet manageable scope.
\end{list2}
Source: \emph{Service‑Oriented Architecture: Analysis and Design for Services and Microservices}, Thomas Erl, 2017

\slide{Chapter 4: Understanding SOA}

\begin{quote}
{\bf Chapter 4: Understanding SOA}\\
This chapter delves into the distinct characteristics and types of service-oriented architecture and further explores the links between the application of the service-orientation design paradigm and technology architecture. The chapter concludes with brief coverage of common SOA project lifecycle stages and organizational roles, with an emphasis on the service inventory analysis, service-oriented analysis, and service-oriented design
phases.
\end{quote}
Source: \emph{Service‑Oriented Architecture: Analysis and Design for Services and Microservices}, Thomas Erl, 2017



\slide{4.2 The Four Common Types of SOA}

To better understand the basic mechanics of SOA, we now need to study the common types of technology architectures that exist within a typical service-oriented
environment:
\begin{list2}
\item \emph{Service Architecture} -- The architecture of a single service.
\item \emph{Service Composition Architecture} -- The architecture of a set of services assembled into a service composition.
\item \emph{Service Inventory Architecture} -- The architecture that supports a collection of related services that are independently standardized and governed.
\item \emph{Service-Oriented Enterprise Architecture} -- The architecture of the enterprise itself, to whatever extent it is service-oriented.
\end{list2}

Source: \emph{Service‑Oriented Architecture: Analysis and Design for Services and Microservices}, Thomas Erl, 2017


\slide{Service-oriented technology architecture}

\hlkimage{12cm}{soabook-4-2-soa-changes.png}

Source: \emph{Service‑Oriented Architecture: Analysis and Design for Services and Microservices}, Thomas Erl, 2017



\slide{SOA Project and Lifecycle Stages}

SOA Adoption Planning
\begin{list2}
\item Scope of planned service inventory and the ultimate target state
\item Milestones representing intermediate target states
\item Timeline for the completion of milestones and the overall adoption effort
\item Available funding and suitable funding model
\item Governance system
\item Management system
\item Methodology
\item Risk assessment
\end{list2}
book then continues with the rest of the SOA Project Stages

Source: \emph{Service‑Oriented Architecture: Analysis and Design for Services and Microservices}, Thomas Erl, 2017


\slide{Chapter 5: Understanding Layers with Services and Microservices}

\begin{quote}
{\bf Chapter 5: Understanding Layers with Services and Microservices}\\
This chapter provides an updated version of the standard service models and corresponding service layers. It incorporates this new content into a new service definition process with the addition of the microservice model and micro task service layer. The relevance of service deployment bundles and containerization are also briefly mentioned in relation to microservice implementation requirements.
\end{quote}
Source: \emph{Service‑Oriented Architecture: Analysis and Design for Services and Microservices}, Thomas Erl, 2017



\slide{Common Service Models}

\begin{list2}
\item \emph{Task Service} -- A service with a non-agnostic functional context that generally corresponds to single-purpose, parent business process logic. A task service will usually encapsulate the composition logic required to compose several other services to complete its task.
\item \emph{Microservice} -- A non-agnostic service often with a small functional scope encompassing logic with specific processing and implementation requirements. Microservice logic is typically not reusable but can have intra-solution reuse potential. The nature of the logic may vary.
\item \emph{Entity Service} -- A reusable service with an agnostic functional context associated with one or more related business entities (such as invoice, customer, or claim). For example, a Purchase Order service has a functional context associated with the processing of purchase order-related data and logic.
\item \emph{Utility Service} -- Although a reusable service with an agnostic functional context as well, this type of service is intentionally not derived from business analysis specifications and models. It encapsulates low-level technology-centric functions, such as notification, logging, and security processing.
\end{list2}

Source: \emph{Service‑Oriented Architecture: Analysis and Design for Services and Microservices}, Thomas Erl, 2017

\slide{End result: functional decomposing large problems}

\hlkimage{15cm}{soabook-5-11-task-service.png}

Source: \emph{Service‑Oriented Architecture: Analysis and Design for Services and Microservices}, Thomas Erl, 2017


\slide{Common objective of all service-orientation design principles}

\hlkimage{15cm}{soabook-5-14-objective.png}

% Source: \emph{Service‑Oriented Architecture: Analysis and Design for Services and Microservices}, Thomas Erl, 2017




\slide{Using containers for microservices}

\hlkimage{15cm}{soabook-5-18-container.png}

Source: \emph{Service‑Oriented Architecture: Analysis and Design for Services and Microservices}, Thomas Erl, 2017

\exercise{ex:postgresql-tutorial}

\exercise{ex:why-soa}

\exercise{ex:cloud-intro}

\exercise{ex:azure-secure-app}

\exercise{ex:download-microservices-book}




\slidenext

\end{document}
