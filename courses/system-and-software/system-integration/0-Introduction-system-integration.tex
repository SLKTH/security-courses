\documentclass[Screen16to9,17pt]{foils}
\usepackage{zencurity-slides}
\externaldocument{system-integration-exercises}
\selectlanguage{english}

% Systemintegration

\begin{document}

\mytitlepage
{0. Introduction}
{KEA System Integration F2021 10 ECTS}

\slide{Course contacts}

\begin{list2}
\item Morten Voetmann Christiansen, morc@kea.dk
\item Henrik Kramselund Jereminsen, independent network and security consultant
\item Email: hkj@zencurity.dk xhek@kea.dk Mobile: +45 2026 6000
\end{list2}

You are welcome to send emails

\slide{Plan for today}

\begin{list2}
\item Create a good starting point for learning
\item Introduce lecturer and students
\item Expectations for this course
\item Literature list walkthrough
%\item Prepare tools for the exercises
%\item Kali and Debian Linux introduction
\end{list2}

%Exercises
%\begin{list2}
%\item Kali Linux installation
%\item Debian Linux installation
%\end{list2}
%Linux is a toolbox we will use and participants will use virtual machines

\slide{Course Materials}

\begin{list1}
\item The materials needed for this course are in multiple parts:
\begin{list2}
%\item Introduktionsmateriale med baggrundsinformation
\item Slide shows - presentation - like this file
\item Exercises - PDF which is updated along the way
\item Books
\item Additional resources from the internet
\end{list2}
\item Note: the presentation slides are not a substitute for reading the books, papers\\ and doing exercises, many details are not shown
\end{list1}

\vskip 1cm

\centerline{I like to use Github for materials, and they are open source}




\slide{Course Data}

{\Large\bf Course: System Integration 10 ECTS}

Teaching dates: Mondays starting 08:15

\begin{list2}
\item Most weeks: 08:15 - 11:30 and 12:15 - 13:45
%\item Week 10 exception: 08:30 - 11:30 and 12:30 - 14:00, then "Praktikinfo" for some of you
\item Watch out for other exceptions, like easter!
\item Exceptions - no teaching on these dates: March 29., April 5., May 24.
\end{list2}

In total 14 days

Semester plan\\
\url{https://zencurity.gitbook.io/kea-it-sikkerhed/system-integration/lektionsplan}

Exam: Online exam: xxx 2020


\slide{Prerequisites}

\hlkimage{12cm}{compare-osi-ip.png}

Participants are expected to have a basic understanding of software architectures and networking, as well as sufficient
programming skills for independent development of software applications.



\slide{Intended Learning Outcomes}

\begin{list1}
\item To get acquainted with the challenges of developing business applications
\item To understand the difference between
\begin{list2}
\item tightly coupled and loosely coupled system
\item synchronous and asynchronous integration
\end{list2}
\item To get an overview of existing technologies and solutions in system integration
\item To get programming practice in developing P2P integration using networking
protocols
\end{list1}

\slide{Course Description}

From: STUDIEORDNING

{\bf Knowledge}

The objective is to give the student knowledge of

\begin{list2}
\item business considerations associated with system integration
\item standards and standardization organizations
\item storage, transformation and integration of data resources
\item techniques used in data conversion and migration
\item the service concept and understanding of its connection with service-oriented architecture
\item technologies that can be used to implement a service-oriented architecture
\item integration tools
\end{list2}

\slide{Skills}

{\bf Skills}

The objective is that the students acquire the ability to

\begin{list2}
\item use object-oriented system in service-oriented architecture
\item design a system for easy integration with other systems and using existing services
\item transform or expand a system, so that it can work in a service-oriented architecture
\item apply patterns that support system integration
\item develop supplementary modules for generic systems
\item integrate generic and other systems
\item choose from different methods of integration
\item translate elements of a business strategy into concrete requirements for system integration
\end{list2}

\slide{Proficiencies}

{\bf Proficiencies}

The objective is that the students acquired proficiency in

\begin{list2}
\item choosing from different integration techniques
\item acquiring knowledge about development in standards for integration
\item adapting IT architecture so that future integration of systems is taken into account
\item converting elements in a business strategy to specific requirements for systems integration
\item adapting a system development method, so that it supports system integration
\end{list2}


\slide{Deliverables and Exam Procedure}

\begin{list2}
\item The course ends with a successful examination. The exam is individual, oral, censored, graded.
\item The duration of the exam is up to 30 minutes.
\item At the exam students can expect being asked any questions related to the learning objectives and presented material.
\end{list2}

Pre-conditions\\
Students need to fulfill certain requirements  completed mandatory tasks - to qualify for participating in the exam.
Fulfilling the requirements automatically signs the student up for an exam. Alternatively, failing in delivering a mandatory
task on time prevents the student from taking part in the exam.

\begin{list2}
\item Deliverables:
\item 2 Mandatory assignments which can be team work up to 3 students
\item Both mandatory assignments are required in order to be entitled to the exam
\end{list2}



\slide{Primary literature}

\hlkrightpic{4cm}{0cm}{old_book_lumen_design_st_01.png}
Primary literature:
\begin{list2}
\item \emph{Enterprise Integration Patterns}, Gregor Hohpe and Bobby Woolf, 2004\\
ISBN: 978-0-321-20068-6 EIP for short
\item \emph{Camel in action}, Claus Ibsen and Jonathan Anstey, 2018\\
ISBN: 978-1-61729-293-4
\item \emph{Service‑Oriented Architecture: Analysis and Design for Services and Microservices},\\ Thomas Erl, 2017
ISBN: 978-0-13-385858-7
\end{list2}
Supporting literature:
\begin{list2}
\item Various internet resources, to be decided
\end{list2}


\slide{Book: Enterprise Integration Patterns}

\hlkimage{6cm}{eip-book.png}

\emph{Enterprise Integration Patterns}, Gregor Hohpe and Bobby Woolf, 2004\\
ISBN: 978-0-321-20068-6 EIP for short

\slide{Companion Web Site}


\begin{quote}
"That's why Bobby Woolf and I documented a pattern language consisting of 65 integration patterns to establish a technology-independent vocabulary and a visual notation to design and document integration solutions. Each pattern not only presents a proven solution to a recurring problem, but also documents common "gotchas" and design considerations.

The patterns are brought to life with examples implemented in messaging technologies, such as JMS, SOAP, MSMQ, .NET, and other EAI Tools. The solutions are relevant for a wide range of integration tools and platforms, such as IBM WebSphere MQ, TIBCO, Vitria, WebMethods (Software AG), or Microsoft BizTalk, messaging systems, such as JMS, WCF, Rabbit MQ, or MSMQ, ESB's such as Apache Camel, Mule, WSO2, Oracle Service Bus, Open ESB, SonicMQ, Fiorano or Fuse ServiceMix."
\end{quote}

Source:\\
\link{https://www.enterpriseintegrationpatterns.com/}

\slide{Book: Camel in Action}

\hlkimage{5cm}{Ibsen-Camel-2ed-HI.png}

\emph{Camel in action}, Claus Ibsen and Jonathan Anstey, 2018\\
ISBN: 978-1-61729-293-4


\slide{Book: Service‑Oriented Architecture}

\hlkimage{5cm}{thomas-erl-book.png}
\emph{Service‑Oriented Architecture: Analysis and Design for Services and Microservices},\\ Thomas Erl, 2017
ISBN: 978-0-13-385858-7

%%% Break?

\slide{Lab Setup}

\hlkimage{6cm}{hacklab-1.png}

\begin{list2}
\item It will be great to have virtualisation running, to try various systems
\item Hardware: modern laptop CPU with virtualisation\\
Dont forget to enable hardware virtualisation in the BIOS
\item Virtualisation software: VMware, Virtual box, HyperV pick your poison
%\item Hackersoftware: Kali Virtual Machine amd64 64-bit\link{https://www.kali.org/}
\item Linux server system: Debian 10 Buster amd64 64-bit\link{https://www.debian.org/}
\item Setup instructions can be found at \link{https://github.com/kramse/kramse-labs}
\end{list2}

\centerline{It is enough if these VMs are pr team}



\slide{Command prompt}

We will use Unix/Linux systems, and you need to use the command line a bit:

\begin{alltt}
\small
[hlk@fischer hlk]$ id
uid=6000(hlk) gid=20(staff) groups=20(staff),
0(wheel), 80(admin), 160(cvs)
[hlk@fischer hlk]$

[root@fischer hlk]# id
uid=0(root) gid=0(wheel) groups=0(wheel), 1(daemon),
2(kmem), 3(sys), 4(tty), 5(operator), 20(staff),
31(guest), 80(admin)
[root@fischer hlk]#
\end{alltt}

\begin{list1}
\item \verb+$+ is commonly used for showing a user login, while a \verb+#+ is for root logins
\item Change from user to root using the command \verb+sudo+ like \verb+sudo -s+
\end{list1}

\slide{Command Syntax}

A common syntax for commands are described like this:
\begin{alltt}
echo [-n] [string ...]
\end{alltt}

\begin{list2}
\item The command is the first thing on the command line, you cannot write \verb+henrik echo+
\item Options are prefixed with dash \verb+-n+, optional ones are in brackets  \verb+[]+
\item Multiple options can be combined into one group like, \verb+tar -cvf+ eller \verb+tar cvf+
\item Some options require arguments, like \verb+tar -cf filename+ where \verb+-f+ needs a filename
\end{list2}



\slide{Manual System}

\hlkimage{7cm}{images/unix-command-1.pdf}

\begin{quote}
 It is a book about a Spanish guy called Manual. You should read it.
       -- Dilbert
\end{quote}

\begin{list1}
\item The Unix/Linux manual system is where you find the options, commands and file formats
\item Manuals must be installed, if not install them immediately
\item Very similar across Unix variants, OpenBSD is known for having an excellent manual pages
\item \verb+man -k+ allows keyword search similar can be done using \verb+apropos+
\end{list1}

Try \verb+man crontab+ and \verb+man 5 crontab+



\slide{Example Manual Page}

\begin{alltt}\footnotesize
\small
NAME
     cal - displays a calendar
SYNOPSIS
     cal [-jy] [[month]  year]
DESCRIPTION
   cal displays a simple calendar.  If arguments are not specified, the cur-
   rent month is displayed.  The options are as follows:
   -j      Display julian dates (days one-based, numbered from January 1).
   -y      Display a calendar for the current year.

The Gregorian Reformation is assumed to have occurred in 1752 on the 3rd
of September.  By this time, most countries had recognized the reforma-
tion (although a few did not recognize it until the early 1900's.)  Ten
days following that date were eliminated by the reformation, so the cal-
endar for that month is a bit unusual.
\end{alltt}

\slide{Unix Command Line Shells}


  \begin{list2}
    \item sh - Bourne Shell
\item bash - Bourne Again Shell, often the default in Linux
\item ksh - Korn shell, originally by David Korn, popular version \verb+pdksh+ public domain ksh
\item csh - C shell, syntax close to the C programming language
\item multiple others exist: zsh, tcsh
  \end{list2}
\begin{list1}
\item Comparable to command.com, cmd.exe and powershell in Windows
\item Also commonly used for small programs, scripts
\item When writing scripts use the characters number sign and exclamation mark (\verb+#!+) in the beginning
\end{list1}

See more in \url{https://en.wikipedia.org/wiki/Shell_(computing)}\\
\url{https://en.wikipedia.org/wiki/Shebang_(Unix)}

\slide{Linux file system and konfiguration}

.
\hlkrightpic{8cm}{0cm}{unix-vfs.pdf}
\begin{list2}

\item Unix/Linux uses a virtual filesystem\\
\url{https://en.wikipedia.org/wiki/Unix_filesystem}
\item No drive letters, just disks mounted in a common tree
\item Everything starts with the file system root \verb+/+ - forward
\item An important directory is \verb+/etc/+ which includes a lot of configuration for the system and applications
\end{list2}



\slide{Part II}

\slide{Definition}

\begin{quote}
  System integration is defined in engineering as the process of bringing together the component sub-systems into one system (an aggregation of subsystems cooperating so that the system is able to deliver the overarching functionality) and ensuring that the subsystems function together as a system,[1] and in information technology[2] as the process of linking together different computing systems and software applications physically or functionally,[3] to act as a coordinated whole.

  The system integrator integrates discrete systems utilizing a variety of techniques such as computer networking, enterprise application integration, business process management or manual programming.[4]
\end{quote}

Source:\\
\url{https://en.wikipedia.org/wiki/System_integration}


\slide{What is Infrastructure}


\hlkimage{10cm}{alexander-schimmeck-SeeM4AnkEHE-unsplash.jpg}

\begin{list2}
\item Enterprises today have a lot of computing systems supporting the business needs
\item These are very diverse and often discrete systems
\end{list2}

\hfill Photo by Alexander Schimmeck on Unsplash

\slide{System Integration challenges}

\hlkimage{5cm}{jorik-kleen-fcLKwDLp2RY-unsplash.jpg}

\begin{list2}
\item Business Challenges
\item Software Challenges
\item Developers Challenges \hfill Photo by Jorik Kleen on Unsplash
\end{list2}




\slide{Business Challenges}

\hlkimage{7cm}{adam-bignell-9tI2z5VZIZg-unsplash.jpg}

\begin{list2}
\item Accumulation of software
\item Legacy systems
\item Partners
\item Various types of data
\item Employee churn, replacement \hfill Photo by Adam Bignell on Unsplash
\end{list2}



\slide{Software Challenges}

\hlkimage{7cm}{john-barkiple-l090uFWoPaI-unsplash.jpg}

\begin{list2}
\item Complexity
\item Various languages
\item Various programming paradigms, client server, monolith, Model View Controller
\item Conflicting data types and available structures
\item Steam train vs electric train \hfill Photo by John Barkiple on Unsplash

\end{list2}




\slide{Developers Challenges}

\hlkimage{10cm}{kelly-sikkema-YK0HPwWDJ1I-unsplash.jpg}

\begin{list2}
\item Work in teams across organisation - and partners, vendors, sub-contractors
\item Work with legacy systems, old technology
\item Learn new Technologies \hfill Photo by Kelly Sikkema on Unsplash
\end{list2}



\slide{Integration Challenges}

% hands

\hlkimage{10cm}{thomas-drouault-IBUcu_9vXJc-unsplash.jpg}

\begin{list2}
\item Enable communication between components
\item Need mediator, interpreter, translator
\item Recognize standard patterns \hfill Photo by Thomas Drouault on Unsplash
\end{list2}




\slide{Technologies used in this course}

The following tools and environments are examples that may be introduced in this course:

\begin{list2}
\item Programming languages and frameworks Java, Spring, Python
\item Development environments IDE NetBeans / Eclipse / IntelliJ, Atom
\item Systems for running Java: TomCat / GlassFish
\item Networking and network protocols: TCP/IP, HTTP, DNS
\item Formats XML, JSON, WSDL, GRPC, msgpack, protobuf, apache thrift
\item Web technologies and services: REST, API, HTML5, CSS
%\item BPL, UML
\item Tools like cURL, Git and Github
\item Integration tools Camel
\item Message queueing systems: MQ
\item Aggregated example platforms: Elastic stack, logstash, elasticsearch, kibana, grafana
\item Cloud and virtualisation Docker, Kubernetes, Azure, AWS, microservices
\end{list2}

\centerline{This list is not complete or a promise }



\slide{OSI and Internet}

\hlkimage{11cm,angle=90}{images/compare-osi-ip.pdf}

\slide{Networking in TCP/IP}

\hlkimage{10cm}{arp-basic.pdf}

\begin{list2}
\item Everything uses TCP/IP today, more or less.
\item Clients make requests, receives responses
\item HyperText Transfer Protocol (HTTP) is an example
\end{list2}



\slide{UDP User Datagram Protocol}
\hlkimage{14cm}{udp-1.pdf}
\begin{list1}
\item Connection-less RFC-768, \emph{connection-less}
\item Used for Domain Name Service (DNS)
\end{list1}

\slide{TCP Transmission Control Protocol}
\hlkimage{12cm}{tcp-1.pdf}

\begin{list1}
\item Connection oriented RFC-791 September 1981, \emph{connection-oriented}
\item Either data delivered in correct order, no data missing, checksum or an error is reported
\item Used for HTTP and others
\end{list1}

\slide{TCP three way handshake}

\hlkimage{6cm}{images/tcp-three-way.pdf}

\begin{list2}
\item Session setup is used in some protocols
\item Other protocols like HTTP/2 can perform request in the first packet
\end{list2}

\slide{Well-known port numbers}

\hlkimage{6cm}{iana1.jpg}

\begin{list1}
\item IANA maintains a list of magical numbers in TCP/IP
\item Lists of protocol numbers, port numbers etc.
\item A few notable examples:
\begin{list2}
\item Port 25/tcp Simple Mail Transfer Protocol (SMTP)
\item Port 53/udp and 53/tcp Domain Name System (DNS)
\item Port 80/tcp Hyper Text Transfer Protocol (HTTP)
\item Port 443/tcp HTTP over TLS/SSL (HTTPS)
\end{list2}
\item Source: \link{http://www.iana.org}
\end{list1}

\slide{DNS root servers}

\hlkimage{20cm}{root-servers.png}

\link{http://root-servers.org/}

\slide{IPv4 packets - header - RFC-791}

\begin{alltt}\small
    0                   1                   2                   3
    0 1 2 3 4 5 6 7 8 9 0 1 2 3 4 5 6 7 8 9 0 1 2 3 4 5 6 7 8 9 0 1
   +-+-+-+-+-+-+-+-+-+-+-+-+-+-+-+-+-+-+-+-+-+-+-+-+-+-+-+-+-+-+-+-+
   |Version|  IHL  |Type of Service|          Total Length         |
   +-+-+-+-+-+-+-+-+-+-+-+-+-+-+-+-+-+-+-+-+-+-+-+-+-+-+-+-+-+-+-+-+
   |         Identification        |Flags|      Fragment Offset    |
   +-+-+-+-+-+-+-+-+-+-+-+-+-+-+-+-+-+-+-+-+-+-+-+-+-+-+-+-+-+-+-+-+
   |  Time to Live |    Protocol   |         Header Checksum       |
   +-+-+-+-+-+-+-+-+-+-+-+-+-+-+-+-+-+-+-+-+-+-+-+-+-+-+-+-+-+-+-+-+
   |                       Source Address                          |
   +-+-+-+-+-+-+-+-+-+-+-+-+-+-+-+-+-+-+-+-+-+-+-+-+-+-+-+-+-+-+-+-+
   |                    Destination Address                        |
   +-+-+-+-+-+-+-+-+-+-+-+-+-+-+-+-+-+-+-+-+-+-+-+-+-+-+-+-+-+-+-+-+
   |                    Options                    |    Padding    |
   +-+-+-+-+-+-+-+-+-+-+-+-+-+-+-+-+-+-+-+-+-+-+-+-+-+-+-+-+-+-+-+-+

                    Example Internet Datagram Header
\end{alltt}

\slide{Exercise: Communicate with HTTP}

Try this - use netcat/ncat, available in Nmap package from \url{Nmap.org}:
\begin{alltt}
\small
$ {\bf netcat www.zencurity.com 80
GET / HTTP/1.0}

HTTP/1.1 200 OK
Server: nginx
Date: Sat, 01 Feb 2020 20:30:06 GMT
Content-Type: text/html
Content-Length: 0
Last-Modified: Thu, 04 Jan 2018 15:03:08 GMT
Connection: close
ETag: "5a4e422c-0"
Referrer-Policy: no-referrer
Accept-Ranges: bytes
...
\end{alltt}





\exercise{ex:sys-int-dateformats}

\exercise{ex:grokdebugger1}

\exercise{gettingstartedelastic}

\exercise{ex:basicDebianVM}

\slidenext{Recommend buying the books!}



\end{document}
