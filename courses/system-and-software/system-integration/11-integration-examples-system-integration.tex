\documentclass[Screen16to9,17pt]{foils}
\usepackage{zencurity-slides}
\externaldocument{system-integration-exercises}
\selectlanguage{english}

% Systemintegration

\begin{document}

\mytitlepage
{11. Integration examples and standards}
{KEA System Integration F2020 10 ECTS}


\slide{This weeks Agenda in system integration}

\begin{list2}
\item Follow the plan:\\
\url{https://zencurity.gitbook.io/kea-it-sikkerhed/system-integration/lektionsplan}
\item Plan for May 18.\\
I will go through the subjects from the books
\end{list2}

\slide{Goals for today}

\hlkimage{6cm}{thomas-galler-hZ3uF1-z2Qc-unsplash.jpg}

Todays goals:
\begin{list2}
\item Finish Enterprise Integration Patterns book
\item Talk about securing, running and deploying Camel
\item Discuss how to manage and monitor production environment
\end{list2}

Photo by Thomas Galler on Unsplash


\slide{Time schedule}
\begin{list2}
\item 08:30 2x 45 min with 10min break\\
EIP Chapter 13: Case Study: Bond Trading System
Chapter 14: Concluding Remarks\\
Finish Enterprise Integration Patterns
\item 10:15 2x 45 min with 10min break\\
Camel chapter 14: Securing Camel\\
Chapter 15: Running and deploying Camel
\item 12:30 2x 45 min with 10min break \\
Camel chapter 16: Management and Monitoring
\item 14:15 45 min\\
Exercises in course so far. Repeat Ansible, how to setup production systems
\end{list2}




\slide{Plan for today}

\begin{list2}
\item Integration examples and standards
\item Running Camel integration Management, Logs, Monitoring, Security
\item - using systems we have used in this course as examples
\end{list2}

Exercises
\begin{list2}
\item Exercises in course so far.
\item Repeat Ansible
\item How to setup production systems
\end{list2}



\slide{Reading Summary}

\begin{list1}
\item EIP 13-14
\item Camel ch 14-16
\item SOA Appendix A
\end{list1}

\slide{EIP chapter 13: Integration Patterns in Practice}

This chapter covers
\begin{list2}
\item Case Study: Bond Trading System
\item Architecture with Patterns
\item Problem Solving With Patterns
\end{list2}

Source: {\footnotesize\\
\emph{Enterprise Integration Patterns}, Gregor Hohpe and Bobby Woolf, 2004\\
ISBN: 978-0-321-20068-6}

Note: Chapter content is available at:\\
\link{https://www.enterpriseintegrationpatterns.com/patterns/messaging/BondTradingCaseStudy.html}

\slide{EIP chapter 14: Concluding Remarks}

This chapter covers
\begin{list2}
\item Finishing up the book
\item List various standard groups
\end{list2}

Source: {\footnotesize\\
\emph{Enterprise Integration Patterns}, Gregor Hohpe and Bobby Woolf, 2004\\
ISBN: 978-0-321-20068-6}

Note: Chapter content is available at:\\
\link{https://www.enterpriseintegrationpatterns.com/patterns/messaging/Future.html}



\slide{Camel chapter 14: Securing Camel}

This chapter covers
\begin{list2}
\item Securing your Camel configuration
\item Web service security
\item Transport security
\item Encryption and decryption
\item Signing messages
\item Authentication and authorization
\end{list2}

Source: {\footnotesize\\
\emph{Camel in action}, Claus Ibsen and Jonathan Anstey, 2018, 2nd edition
ISBN: 978-1-61729-293-4}

\slide{Camel chapter 15: Running and deploying Camel}

This chapter covers

\begin{list2}
\item Starting and stopping Camel safely
\item Adding and removing routes at runtime
\item Deploying Camel
\item Running standalone
\item Running in web containers
\item Running in Java EE servers
\item Running with OSGi
\item Running with CDI
\end{list2}

Source: {\footnotesize\\
\emph{Camel in action}, Claus Ibsen and Jonathan Anstey, 2018, 2nd edition
ISBN: 978-1-61729-293-4}

\slide{Camel chapter 16: Management and Monitoring}

This chapter covers
\begin{list2}
\item Monitoring Camel instances
\item Tracking application activities
\item Using notifications
\item Managing Camel applications with JMX and REST
\item Understanding and using the Camel management API
\item Gathering runtime performance statistics
\item Using Dropwizard metrics with Camel
\item Developing custom components for management
\end{list2}

Source: {\footnotesize\\
\emph{Camel in action}, Claus Ibsen and Jonathan Anstey, 2018, 2nd edition
ISBN: 978-1-61729-293-4}



\slide{SOA Appendix A: Service-Orientation Principles Reference}

\begin{quote}
This appendix provides profile tables for the patterns referenced throughout this
book. As explained in Chapter 1, each pattern reference is suffixed with the page
number of its corresponding profile table in this appendix.
\end{quote}

Similar to Appendix B: SOA Design Patterns Reference

Source: {\footnotesize\\
\emph{Service‑Oriented Architecture: Analysis and Design for Services and Microservices},\\ Thomas Erl, 2017
ISBN: 978-0-13-385858-7}

\slide{Design patterns are helpful}

Design patterns are helpful because they:
\begin{list2}
\item Represent field-tested solutions to common design problems
\item Organize design intelligence into a standardized and easily “referenceable” format
\item Are generally repeatable by most IT professionals involved with design
\item Can be used to ensure consistency in how systems are designed and built
\item Can become the basis for design standards
\item Are usually fl exible and optional (and openly document the impacts of their appli-
cation and even suggest alternative approaches)
\item Can be used as educational aids by documenting specific aspects of system design
(regardless of whether they are applied)
\item Can sometimes be applied prior and subsequent to the implementation of a system
\item Can be supported via the application of other design patterns that are part of the
same collection
\item Enrich the vocabulary of a given IT field because each pattern is given a
meaningful name
\end{list2}


\slide{Part I 08:30 2x 45 min}

EIP Chapter 13 and 14
Finish Enterprise Integration Patterns

\slide{Case Study: Bond Trading System}

This chapter covers
\begin{list2}
\item Case Study: Bond Trading System
\item Architecture with Patterns
\item Problem Solving With Patterns
\end{list2}

Source: {\footnotesize\\
\emph{Enterprise Integration Patterns}, Gregor Hohpe and Bobby Woolf, 2004\\
ISBN: 978-0-321-20068-6}

Note: We will now continue at the book site:\\
{\footnotesize\link{https://www.enterpriseintegrationpatterns.com/patterns/messaging/BondTradingCaseStudy.html}}

\slide{Concluding Remarks}


\begin{list2}
\item Emerging Standards and Futures in Enterprise Integration
\end{list2}

Note: We will now continue at the book site:\\
{\footnotesize\link{https://www.enterpriseintegrationpatterns.com/patterns/messaging/Future.html}}


\slide{Further patterns}

\begin{quote}

\end{quote}

\begin{list2}
  \item Further patterns, newer data and articles are available from the authors:\\
  \link{https://www.enterpriseintegrationpatterns.com/ramblings.html}
\end{list2}


\slide{Modern Examples for Enterprise Integration Patterns}

\hlkimage{10cm}{eip-modern-examples.png}

Modern Examples for Enterprise Integration Patterns\\
{\footnotesize\link{https://www.enterpriseintegrationpatterns.com/ramblings/eip1_examples_updated.html}}



\slide{What Products Implement or Use Enterprise Integration Patterns?}




The patterns are not tied to a specific implementation. They help you design better solutions, whether you use any of the following platforms:
\begin{list2}
  \item {\bf EAI and SOA platforms}, such as IBM WebSphere MQ, TIBCO, Vitria, Oracle Service Bus, WebMethods (now Software AG), Microsoft BizTalk, or Fiorano.
\item {\bf Open source ESB's} like Mule ESB, JBoss Fuse, Open ESB, WSo2, Spring Integration, or Talend ESB
\item {\bf Message Brokers} like ActiveMQ, Apache Kafka, or RabbitMQ
\item {\bf Web service- or REST-based integration}, including Amazon Simple Queue Service (SQS) or Google Cloud Pub/Sub
\item {\bf JMS-based messaging systems}
\item {\bf Microsoft technologies} like MSMQ or Windows Communication Foundation (WCF)
\end{list2}

\slide{Stencils for EIP}

\hlkimage{8cm}{eip-stencils.png}

\begin{quote}
Document. You can create design documents using our icon language by downloading the Visio stencil or using the OmniGraffle stencil created by one of our readers.
\end{quote}

\begin{list2}
\item \link{https://www.enterpriseintegrationpatterns.com/patterns/messaging/downloads.html}
\item \link{http://www.graffletopia.com/stencils/137}
\end{list2}

\slide{Part II 10:15 2x 45 min}

Camel chapter 14-15

\slide{Good security}

\hlkimage{15cm}{god-sikkerhed.pdf}

\begin{list1}
\item You always have limited resources for protection - use them as best as possible
\end{list1}


\slide{Recommendations}

\begin{list1}
\item {\bf Keep updated!}\\ - read web sites, books, articles, mailing lists, Twitter, ...
\item {\bf Always have a chapter on security evaluation }\\ - any process must have security, like RFC Request for Comments have
\item {\bf Incident Response}\\ - you WILL have security incidents, be prepared
\item {\bf Write down security policy}\\ - including software and e-mail policies
\end{list1}

\slide{Advice}

\begin{list1}
\item Use technology
\item Learn the technology - read the freaking manual
\item Think about the data you have, upload, facebook license?! WTF!
\item Think about the data you create - nude pictures taken, where will they show up?
\begin{list2}
\item Turn off features you don't use
\item Turn off network connections when not in use
\item Update software and applications
\item Turn on encryption: IMAP{\bf S}, POP3{\bf S},
  HTTP{\bf S} also for data at rest, full disk encryption, tablet encryption
\item Lock devices automatically when not used for 10 minutes
\item Dont trust fancy logins like fingerprint scanner or face recognition on cheap devices
\end{list2}
\end{list1}

But which features to disable? Let the security principles guide you

\slide{Confidentiality, Integrity and Availability}

\hlkimage{8cm}{cia-triad-uk.pdf}

\begin{list1}
\item We want to protect something
\item Confidentiality - data kept a secret
\item Integrity - data is not subjected to unauthorized changes
\item Availability - data and systems are available when needed
\end{list1}


\slide{Security is a process}

\begin{list1}
\item Remember:
\begin{list2}
\item what is information and security?
\item Data kept electronically
\item Data kept in physical form
\item Dont forget the human element of security
\end{list2}
\item Incident Response and Computer Forensics reaction to incidents
\item Good security is the result of planning and long-term work
\end{list1}
\vskip 1cm
\centerline{\color{titlecolor}\LARGE Security is a process, not a product, Bruce Schneier}

Source for quote: \link{https://www.schneier.com/essays/archives/2000/04/the_process_of_secur.html}


\slide{Work together}

\hlkimage{9cm}{Shaking-hands_web.jpg}

\begin{list1}
\item Team up!
\item We need to share security information freely
\item We often face the same threats, so we can work on solving these together
\end{list1}

\slide{Goals of Security}

\begin{list1}
\item Prevention - means that an attack will fail
\item Detection - determine if attack is underway, or has occured - report it
\item Recovery - stop attack, assess damage, repair damage
\end{list1}

\slide{Policy and Mechanism}

\begin{quote}
{\bf Definition 1-1.} A \emph{security policy} is a statement of what is, and what is not, allowed.

{\bf Definition 1-2.} A \emph{security mechanism} is a method, tool or procedure for enforcing a security policy.
\end{quote}

Quote from Matt Bishop, Computer Security section 1.3


\slide{Balanced security}

\hlkimage{21cm}{afbalanceret-sikkerhed.pdf}

\begin{list1}
\item Better to have the same level of security
\item If you have bad security in some part - guess where attackers will end up
\item Hackers are not required to take the hardest path into the network
\item Realize there is no such thing as 100\% security
\end{list1}


\slide{Cost-Benefit Analysis}
% ROI?

\begin{list1}
\item Benefits of computer security must be weighed against value of assets
\item Often more expensive to add security mechanisms to a system, than designing them in
\end{list1}


\slide{Risk management defined}

\hlkimage{18cm}{shon-harris-risk-management.png}

Source: Shon Harris \emph{CISSP All-in-One Exam Guide}



\slide{Securing your Camel configuration}


\begin{alltt}\footnotesize
  [janstey@bender]$ cd apache-­camel-2.20.1/
  [janstey@bender]$ wget http://repo1.maven.org/maven2/org/jasypt/jasypt/1.9.2/
  jasypt-1.9.2.jar
  [janstey@bender]$ java -cp jasypt-1.9.2.jar:lib/camel-jasypt-2.20.1.jar org.
  apache.camel.component.jasypt.Main -help
  Apache Camel Jasypt takes the following options

  -help = Displays the help screen
  -command <command> = Command either encrypt or decrypt
  -password <password> = Password to use
  -input <input> = Text to encrypt or decrypt
  -algorithm <algorithm> = Optional algorithm to use
\end{alltt}

\begin{list2}
\item Encrypting configuration is possible using \verb+camel-­jasypt+
\item Maybe not recommended way to do this now
\end{list2}




\slide{Web service security}


\begin{quote}
  Web services are an extremely useful integration technology for distributed applica-
  tions. They are often associated with service-­oriented architecture (SOA), in which
  each service is defined as a web service.
\end{quote}

\begin{list2}
  \item TL;DR Almost everything is a web service today, even Mobile ppps
  \item Chapter is really inadequate for describing real security
  \item Never use unencrypted FTP and HTTP, unfortunately it happens ...
  \item Recommend The Open Web Application Security Project (OWASP) instead \link{https://owasp.org/}
\end{list2}

\slide{Transport security}


\begin{list2}
  \item Transport Security is needed for HTTP, making it HTTPS
  \item In Java, TLS is typically configured using the Java Secure Socket Exten-
sion (JSSE) API provided with the JRE
\end{list2}

\slide{Cryptography}

\begin{list1}
\item Cryptography or cryptology is the practice and study of techniques for secure communication
\item Modern cryptography is heavily based on mathematical theory and computer science practice; cryptographic algorithms are designed around computational hardne
ss assumptions, making such algorithms hard to break in practice by any adversary
\item Symmetric-key cryptography refers to encryption methods in which both the sender and receiver share the same key, to ensure confidentiality, example algorit
hm AES
\item Public-key cryptography (like RSA) uses two related keys, a key pair of a public key and a private key. This allows for easier key exchanges, and can provid
e confidentiality, and methods for signatures and other services
\end{list1}

Source: \link{https://en.wikipedia.org/wiki/Cryptography}


\slide{DES, Triple DES og AES}

\hlkimage{15cm}{images/AES_head.png}

\begin{list1}
\item DES - old and retired!!
\item In 2001 a newer standard was adopted Advanced Encryption Standard (AES)
\item It replaces Data Encryption Standard (DES)
\item Algorithm is Rijndael developed by Joan Daemen og Vincent Rijmen.
\item See \link{https://en.wikipedia.org/wiki/Advanced_Encryption_Standard}
\item Animations can be found (including errors)\\
 \link{https://www.youtube.com/watch?v=mlzxpkdXP58}
\end{list1}

\slide{AES Advanced Encryption Standard}

\hlkimage{10cm}{aes-overview.png}

\begin{list2}
\item The official Rijndael web site displays this image to promote understanding of the Rijndael round transformation [8].
\item Key sizes 128,192,256 bit typical
\item Some extensions in cryptosystems exist: XTS-AES-256 really is 2 instances of AES-128 and 384 is two instances of AES-192 and 512 is two instances of AES-256
\item \link{https://en.wikipedia.org/wiki/RSA_(cryptosystem)}
\end{list2}


\slide{RSA}

\begin{quote}
RSA (Rivest–Shamir–Adleman) is one of the first public-key cryptosystems and is widely used for secure data transmission. ...
In RSA, this asymmetry is based on the practical difficulty of the factorization of the product of two large prime numbers, the "factoring problem". The acronym RSA is made of the initial letters of the surnames of Ron Rivest, Adi Shamir, and Leonard Adleman, who first publicly described the algorithm in 1978.
\end{quote}

\begin{list2}
\item Key sizes 1,024 to 4,096 bit typical
\item  Quote from: \link{https://en.wikipedia.org/wiki/RSA_(cryptosystem)}
\end{list2}


\slide{Diffie Hellman exchange}

{~}
\hlkrightpic{7cm}{-15mm}{800px-Diffie-Hellman_Key_Exchange.png}

\begin{quote}
Diffie–Hellman key exchange (DH)[nb 1] is a method of securely exchanging cryptographic keys over a public channel and was one of the first public-key protocols as originally conceptualized by Ralph Merkle and named after Whitfield Diffie and Martin Hellman.[1][2] DH is one of the earliest practical examples of public key exchange implemented within the field of cryptography.
... The scheme was first published by Whitfield Diffie and Martin Hellman in 1976
\end{quote}

\begin{list2}
\item Quote from: {\small \link{https://en.wikipedia.org/wiki/Diffie-Hellman_key_exchange}}
\item Today we also use elliptic curves with DH \\{\small \link{https://en.wikipedia.org/wiki/Elliptic-curve_cryptography}}
\end{list2}

\slide{Elliptic Curve }

\begin{quote}
Elliptic-curve cryptography (ECC) is an approach to public-key cryptography based on the algebraic structure of elliptic curves over finite fields. ECC requires smaller keys compared to non-EC cryptography (based on plain Galois fields) to provide equivalent security.[1]
\end{quote}

\begin{list2}
\item Today we use \link{https://en.wikipedia.org/wiki/Elliptic-curve_cryptography}
\end{list2}



\slide{Transport Layer Security (TLS)}

\hlkimage{5cm}{crypto-class.png}

\begin{list1}
\item Oprindeligt udviklet af Netscape Communications Inc.
\item Secure Sockets Layer SSL er idag blevet adopteret af IETF og kaldes
derfor også for Transport Layer Security TLS
TLS er baseret på SSL Version 3.0
\item RFC-2246 The TLS Protocol Version 1.0 fra Januar 1999
\item RFC-3207 SMTP STARTTLS
\item Det er svært!
\item Stanford Dan Boneh udgiver en masse omkring crypto\\ \link{https://crypto.stanford.edu/~dabo/cryptobook/}
\end{list1}


\slide{SSL/TLS protocols}
\hlkimage{12cm}{imap-ssl.png}

\begin{list2}
\item Many protocols have been extended to cover TLS
\item HTTPS vs HTTP
\item IMAPS, POP3S, osv.
\item Some use the same port, others use two different ports IMAP 143/tcp vs IMAPS 993/tcp
\item Those using the same port often can use START TLS, like:\\
SMTP STARTTLS RFC-3207
\end{list2}



\slide{Encryption and Transport Security Summary}

%\hlkimage{}{}

\begin{quote}
  Camel is unsecured by default—Probably the most important point is that Camel
  has no security settings turned on by default. This is great for development, but
  before your application is deployed in the real world, you’ll most likely need
  some form of security enabled.
\end{quote}

\begin{list2}
  \item You need to use Transport Layer Security (TLS)
  \item I recommend using Let's Encrypt \link{https://letsencrypt.org/}
\end{list2}

Source: {\footnotesize\\
\emph{Camel in action}, Claus Ibsen and Jonathan Anstey, 2018, 2nd edition
ISBN: 978-1-61729-293-4}



\slide{Chapter 15: Running and deploying Camel}

This chapter covers
\begin{list2}
\item Starting and stopping Camel safely
\item Adding and removing routes at runtime
\item Deploying Camel
\item Running standalone
\item Running in web containers
\item Running in Java EE servers
\item Running with OSGi
\item Running with CDI
\end{list2}


\slide{Deploying Camel}

This section presents five deployment strategies that are possible with Camel and
their strengths and weaknesses:
\begin{list2}
\item Embedding Camel in a Java application
\item Running Camel in a web environment such as Apache Tomcat
\item Running Camel inside WildFly
\item Running Camel in an OSGi container such as Apache Karaf
\item Running Camel in a container that supports CDI, such as Apache Karaf or WildFly
\end{list2}

Source: {\footnotesize\\
\emph{Camel in action}, Claus Ibsen and Jonathan Anstey, 2018, 2nd edition
ISBN: 978-1-61729-293-4}



\slide{Deploying to Apache Tomcat}

%\hlkimage{}{}

\begin{alltt}\footnotesize
  [janstey@ghost apache-­tomcat-8.5.23]$ bin/startup.sh
  Using CATALINA_BASE: /home/janstey/kits/apache-­tomcat-8.5.23
  Using CATALINA_HOME: /home/janstey/kits/apache-­tomcat-8.5.23
  Using CATALINA_TMPDIR: /home/janstey/kits/apache-­tomcat-8.5.23/temp
  Using JRE_HOME:
  /usr/java/jdk1.8.0_91/
  Using CLASSPATH:
  /home/janstey/kits/apache-­tomcat-8.5.23/bin/bootstrap.
  jar:/home/janstey/kits/apache-­tomcat-8.5.23/bin/tomcat-­juli.jar
  Tomcat started.
\end{alltt}

\begin{list2}
  \item First starting Tomcat, we did this early in the course
  \item Then copy war file - Java Archive with Web Application \\
  \verb+cp target/riderautoparts-­war-2.0.0.war ~/kits/apache-­tomcat-8.5.23/webapps/+
\end{list2}





\slide{Part III 12:30 2x 45min}

Camel chapter 16: Management and monitoring


\slide{Monitoring Camel instances}

%\hlkimage{}{}

\begin{list2}
\item \emph{Network level} -- This is the most basic level, where you check that the network connectivity is working.
\item \emph{JVM level} -- At this level, you check the JVM that hosts the Camel application. The JVM exposes a standard set of data using the JMX technology.
\item \emph{Application level} -- Here you check the Camel application using JMX or other
techniques.
\item JConsole can be used for JMX connection
\item Read \url{https://tomcat.apache.org/tomcat-9.0-doc/monitoring.html} about Java Management Extensions (JMX) \url{https://en.wikipedia.org/wiki/Java_Management_Extensions}\\
Note: references to RMI, JMS, SNMP, HTTP etc.
\end{list2}


\slide{Logs: Tracking application activities}


\begin{alltt}
\small
*.err;kern.debug;auth.notice;authpriv.none;mail.crit    /dev/console
*.notice;auth,authpriv,cron,ftp,kern,lpr,mail,user.none /var/log/messages
kern.debug;user.info;syslog.info                        /var/log/messages
auth.info                                               /var/log/authlog
authpriv.debug                                          /var/log/secure
...
# Uncomment to log to a central host named "loghost".
#*.notice;auth,authpriv,cron,ftp,kern,lpr,mail,user.none        @loghost
kern.debug,user.info,syslog.info                               @loghost
auth.info,authpriv.debug,daemon.info                           @loghost
\end{alltt}
\vskip 1cm
\centerline{Centralized syslogging is recommended!}

\slide{Kibana}

\hlkimage{12cm}{kibanascreenshothomepagebannerbigger.jpg}

\centerline{Highly recommended for a lot of data visualisation}

Non-programmers can create, save, and share dashboards

Source:
\link{https://www.elastic.co/products/kibana}



\slide{Logstash pipeline }

\begin{quote}
  Logstash is an open source, server-side data processing pipeline that ingests data from a multitude of sources simultaneously, transforms it, and then sends it
to your favorite “stash.” (Ours is Elasticsearch, naturally.)\\
  \link{https://www.elastic.co/products/logstash}
\end{quote}

\begin{verbatim}
input { stdin { } }
output {
  elasticsearch { host => localhost }
  stdout { codec => rubydebug }
}
\end{verbatim}


\slide{Grok expresssions}

{\footnotesize
\begin{verbatim}
  filter {
    if [type] == "syslog" {
      grok {
        match => { "message" => "%{SYSLOGTIMESTAMP:syslog_timestamp}
        %{SYSLOGHOST:syslog_hostname} %{DATA:syslog_program}
        (?:\[%{POSINT:syslog_pid}\])?: %{GREEDYDATA:syslog_message}" }
        add_field => [ "received_at", "%{@timestamp}" ]
        add_field => [ "received_from", "%{host}" ]
      }
      syslog_pri { }
      date {
        match => [ "syslog_timestamp", "MMM  d HH:mm:ss", "MMM dd HH:mm:ss" ]
      }
    }
  }
\end{verbatim}
}

\begin{list2}
\item Logstash filter expressions grok can normalize and split data into fields
\end{list2}

Source:
Config snippet from recommended link\\
{\small\link{http://logstash.net/docs/1.4.1/tutorials/getting-started-with-logstash}}


\slide{Managing Camel applications}

\begin{list2}
\item \emph{Camel application lifecycle} -- Control your Camel application, such as stopping and
starting routes, and much more using a broad range of ways with JMX, Jolokia,
hawtio, and the ControlBus component
\item \emph{Camel management API} -- Learn about the programming API from Camel that
defines the management API.
\item \emph{Performance statistics} -- Discover which key metrics Camel captures about your
Camel application performance, and how to access these metrics for custom
reporting and hook into monitoring and alert tools.
\item \emph{Management enabling custom components} -- Learn to program your custom Camel
components and Java beans so they’re management enabled out of the box, as if
they were first-­class from the Camel release.
\end{list2}



\slide{Gathering runtime performance statistics}

Book has lots of examples with hawtio and JConsole

You can try this example by running the following Maven goal from the chapter16/
custom-­bean directory:

\verb+mvn compile exec:java+

If you run the example and connect to the JVM using JConsole, you can find
the custom bean in the JMX tree under the Camel processor tree, as shown in figure 16.16.


\slide{Production setup}

%\hlkimage{}{}

\begin{quote}
The Prometheus trait configures the Prometheus JMX exporter and exposes the integration with a Service and a ServiceMonitor resources so that the Prometheus endpoint can be scraped.
\end{quote}

\begin{list2}
  \item I would propose looking into Prometheus instead
  \item \link{https://camel.apache.org/camel-k/latest/traits/prometheus.html}
\end{list2}

\slide{Modern Monitoring}

\begin{list2}
  \item Prometheus collects metrics from monitored targets by scraping metrics HTTP endpoints on these targets. Since Prometheus also exposes data in the same manner about itself, it can also scrape and monitor its own health.\\
  \link{https://prometheus.io/}
  \item Grafana is the open source analytics and monitoring solution for every database\\
  \link{https://grafana.com/}
  \item Loki is a horizontally-scalable, highly-available, multi-tenant log aggregation system inspired by Prometheus. \\
  \link{https://grafana.com/oss/loki/}
\end{list2}

\slide{Part IV 12:30 2x 45min}

Exercises in course so far. Repeat Ansible, how to setup production systems

Let's run some code, repeat or something new. Not required for the exam, but helps understand the systems:
\begin{list2}
\item Repeat the Ansible setup, re-run Kramse Labs Ansible playbooks
\item Repeat Logstash, Elasticsearch and Kibana tasks -- creating a dashboard perhaps
\item Run Tomcat some more, deploy a war
\item Perhaps try \link{https://prometheus.io/docs/prometheus/latest/getting_started/}
\end{list2}

\slidenext

\end{document}
