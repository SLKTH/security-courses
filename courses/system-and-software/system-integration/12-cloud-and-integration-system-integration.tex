\documentclass[Screen16to9,17pt]{foils}
\usepackage{zencurity-slides}
\externaldocument{system-integration-exercises}
\selectlanguage{english}

% Systemintegration

\begin{document}

\mytitlepage
{12. Cloud and Cloud integration}
{KEA System Integration F2020 10 ECTS}


\slide{Goals for today}

\hlkimage{6cm}{thomas-galler-hZ3uF1-z2Qc-unsplash.jpg}

Todays goals:
\begin{list2}
\item Finish the Camel book
\item Talk about cloud systems, in particular Kubernetes
\item Talk about cloud security, as time permits
\end{list2}

Photo by Thomas Galler on Unsplash

\slide{Time schedule}

\begin{list2}
\item 08:15 - 09:00 and
\item 09:15 - 10:00
2x sessions with 15min break\\
Camel chapter 18: Microservices with Docker and Kubernetes
\item 10:15 - 11:30 Kubernetes demo, discussion\\

\item 12:15 - 13:45
Exercises
\end{list2}


\slide{Plan for today}

\begin{list2}
\item Microservices with Docker and Kubernetes
\item Cloud and Cloud integration
\item Running Camel on Docker
\item Getting more into Kubernetes
\end{list2}

Exercises
\begin{list2}
\item Running Java microservices on Docker
\item Getting started with Kubernetes -- run Minikube on the web
\item Research how to run a few applications on Kubernetes
\end{list2}


\slide{Part I}

Microservices with Docker and Kubernetes


\slide{Docker }

%\hlkimage{}{}

\begin{quote}
  Package Software into Standardized Units for Development, Shipment and Deployment
  A container is a standard unit of software that packages up code and all its dependencies so the application runs quickly and reliably from one computing environment to another. A Docker container image is a lightweight, standalone, executable package of software that includes everything needed to run an application: code, runtime, system tools, system libraries and settings.
\end{quote}
Source: \\{\footnotesize
\url{https://www.docker.com/resources/what-container}}

\begin{list2}
  \item One of the most popular deployment methods today
\end{list2}

\slide{Containerized Applications}

\hlkimage{8cm}{container-what-is-container.png}

Source: {\footnotesize
\url{https://www.docker.com/resources/what-container}}

\begin{list2}
  \item See also \link{https://en.wikipedia.org/wiki/Linux_namespaces}\\
   \emph{Various container software use Linux namespaces in combination with cgroups to isolate their processes, including Docker[11] and LXC.}
\end{list2}

\slide{Docker Images and layers}

\hlkimage{10cm}{container-layers.jpg}

\begin{quote}
A Docker image is built up from a series of layers. Each layer represents an instruction in the image’s Dockerfile. Each layer except the very last one is read-only.
\end{quote}

Source: {\footnotesize
\link{https://docs.docker.com/storage/storagedriver/}}



\slide{Kubernetes}

%\hlkimage{}{}

\begin{quote}
  Kubernetes (K8s) is an open-source system for automating deployment, scaling, and management of containerized applications.
  It groups containers that make up an application into logical units for easy management and discovery. Kubernetes builds upon 15 years of experience of running production workloads at Google, combined with best-of-breed ideas and practices from the community.
\end{quote}
Source: {\footnotesize
\link{https://kubernetes.io/}}

Key points:
\begin{list2}
\item Open source originally from Google
\item Scalable
\item Uses containers inside
\item Infrastructure as code
\end{list2}


\slide{Infrastructure as code}

%\hlkimage{}{}

\begin{quote}
{\bf Infrastructure as code (IaC)} is the process of managing and provisioning computer data centers through machine-readable definition files, rather than physical hardware configuration or interactive configuration tools.[1] The IT infrastructure managed by this comprises both physical equipment such as bare-metal servers as well as virtual machines and associated configuration resources. The definitions may be in a version control system. It can use either scripts or declarative definitions, rather than manual processes, but the term is more often used to promote declarative approaches.
\end{quote}
Source: {\footnotesize
\link{https://en.wikipedia.org/wiki/Infrastructure_as_code}}

\begin{list2}
  \item Has become the norm in many places
\end{list2}




\slide{Camel chapter 18: Microservices with Docker and Kubernetes}

This chapter covers
\begin{list2}
\item Running Camel on Docker
\item Getting started with Kubernetes
\item Running and debugging Camel on Kubernetes
\item Understanding essential Kubernetes concepts
\item Building resilient microservices on Kubernetes
\item Testing Camel microservices on Kubernetes
\item Introducing fabric8, Kubernetes Helm, and OpenShift
\end{list2}

Source: {\footnotesize\\
\emph{Camel in action}, Claus Ibsen and Jonathan Anstey, 2018, 2nd edition
ISBN: 978-1-61729-293-4}



\slide{18.1.2 Running Camel on Docker}

%\hlkimage{}{}


  the Dockerfile you’ll use to run the Spring Boot client microservice contains
  just three lines of text:
\begin{alltt}
  FROM openjdk:latest
  COPY maven /maven/
  CMD java -jar maven/spring-­docker-2.0.0.jar
\end{alltt}

  A Docker image is a compressed TAR file that includes the Dockerfile in the root
  alongside other files you want to include in the image. The Spring Boot Docker image
  consists of only two files:
\begin{alltt}
  maven/spring-­docker-2.0.0.jar
  Dockerfile
\end{alltt}

\begin{list2}
  \item We have seen problems with various JDK versions
  \item Running on Docker might be simpler
  \item Help available: {\footnotesize\link{https://docs.docker.com/develop/develop-images/dockerfile_best-practices/}}
\end{list2}



\slide{Getting started with Kubernetes: Minikube}

\centerline{Not running it now, but later}
\begin{alltt}
  minikube start --cpus 2 --memory 2048 --disk-­size 10g
\end{alltt}
The last parameter is important; it specifies which VM driver to use (see Minikube documentation for details). After the installation is complete, you can get the status of Minikube:
\begin{alltt}
  $ minikube status
  minikubeVM: Running
  localkube: Running
  kubectl: Correctly Configured: pointing to minikube-­vm at 192.168.64.2
\end{alltt}

This means the local Kubernetes cluster is up and running.

\begin{list2}
\item We already saw Minikube in our browser
\end{list2}


\slide{Running and debugging Camel on Kubernetes}

%\hlkimage{}{}

\begin{quote} {\bf
18.3 Running Camel and other applications in Kubernetes}
When you {\bf run applications\\
on Kubernetes, they run as containers loaded from Docker images.} The information you learned in the previous section about running Camel on Docker is required knowledge for working with Kubernetes.
\end{quote}


\slide{Understanding essential Kubernetes concepts}

%\hlkimage{}{}

\begin{alltt}
$ kubectl get deployment -o yaml hello-­world
apiVersion: extensions/v1beta1
kind: Deployment
metadata:
  annotations:
    deployment.kubernetes.io/revision: "1"
  creationTimestamp: 2017-11-04T20:33:36Z
  generation: 1
  labels:
    foo: bar
  name: hello-­world
  ...
\end{alltt}

\begin{list2}
\item Example YAML file from Kubernetes
\end{list2}




\slide{Building resilient microservices on Kubernetes}

\hlkimage{18cm}{kubernetes-scaling-up.png}

\begin{list2}
\item Kubernetes can be told to create more pods/containers
\item AND can check if it alive and good
\end{list2}



\slide{Introducing fabric8, Kubernetes Helm, and OpenShift}

%\hlkimage{}{}

\begin{quote}

\end{quote}

Book lists multiple tools that can help making Java applications Kubernetes-ready:
\begin{list2}
\item Docker Maven plugin
\item Kubernetes-ready fabric8 Maven plugin
\item Kubernetes Helm
\item OpenShift
\end{list2}

We wont go into detail with these, and check if better tools are available before use

\slide{Securing Kubernetes}

%\hlkimage{}{}

\begin{quote}
  Attacking and Defending Kubernetes, with Ian Coldwater\\
  Ian Coldwater specializes in breaking and hardening Kubernetes, containers, and cloud native infrastructure. A pre-eminent voice in the Kubernetes security community, Ian is currently a Lead Platform Security Engineer at Heroku. Ian joins Adam Glick and Craig Box to talk about the offensive and defensive arts.
\end{quote}
{\footnotesize\link{https://www.heroku.com/podcasts/kubernetes-podcast-from-google/attacking-and-defending-kubernetes-with-ian-coldwater}}

\begin{list2}
  \item Securing Kubernetes can be hard work
  \item
  \item follow Ian Coldwater, @IanColdwater \link{https://twitter.com/iancoldwater}
\end{list2}



\slide{Helm Database}

%\hlkimage{}{}

\begin{quote}

\end{quote}

\begin{list2}
  \item Book uses Helm to deploy a database
  \item Easier than running Postgresql yourself?
  \item Do you want your database inside Kubernetes? why/why not
\end{list2}

\slide{Similar thoughts about load balancing}

%\hlkimage{}{}

\begin{quote}

\end{quote}

\begin{list2}
  \item Do we run everything inside the Kubernetes cluster?
  \item Do we want/need hardware acceleration for things like load balancing and HTTPS/TLS termination
\end{list2}


\slide{Part II: Running Kubernetes}


\begin{list2}
\item I will start up Kubernetes and demonstrate the setup
\item I will act as if this is part of my job, and I am new to Kubernetes
\end{list2}


\slide{Getting started with Kubernetes: Minikube}

\centerline{Not running it now, but later}
\begin{alltt}\footnotesize
  minikube start --cpus 2 --memory 2048 --disk-­size 10g
\end{alltt}
The last parameter is important; it specifies which VM driver to use (see Minikube documentation for details). After the installation is complete, you can get the status of Minikube:
\begin{alltt}\footnotesize
  $ minikube status
  minikubeVM: Running
  localkube: Running
  kubectl: Correctly Configured: pointing to minikube-­vm at 192.168.64.2
\end{alltt}

This means the local Kubernetes cluster is up and running.

\begin{list2}
\item We already saw Minikube in our browser
\item I will run a local version of Minukube on my Debian
\item We will go through a bit of details with regards to Kubernetes
\end{list2}


\slide{Exercise: Lets run Kubernetes}

\hlkimage{10cm}{kubernetes-minikube.png}

\begin{list2}
  \item \link{https://kubernetes.io/docs/tutorials/hello-minikube/}
\end{list2}

\exercise{ex:minikube-web}

\exercise{ex:k8s-nginx-lb}

\exercise{ex:k8s-postgrsql}

\end{document}
