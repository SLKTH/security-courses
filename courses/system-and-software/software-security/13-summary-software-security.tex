\documentclass[Screen16to9,17pt]{foils}
\usepackage{zencurity-slides}

\externaldocument{communication-and-network-security-exercises}
\selectlanguage{english}

\begin{document}

\mytitlepage
{13. Summary and Exam Preparation}
{KEA Kompetence Computer Systems Security}



\slide{Goals and Plan for today}

\hlkimage{12cm}{chris-benson-nKEARsgmrqc-unsplash.jpg}

\begin{list1}
\item Go through exam reading list, Literature list walkthrough, Subject list walkthrough
\item Trial exam, show how it works \hfill {\small Photo by Chris Benson on Unsplash}

\end{list1}



\slide{Literature list walkthrough}

\begin{list1}
\item Our reading list is at:\\
{\small\url{https://zencurity.gitbook.io/kea-it-sikkerhed/softwaresikkerhed/lektionsplan}}
\item Not all are required reading for the exam!
\item We will now go through the list and comment, ask questions
\item Selection criteria and goals:
\begin{list2}
\item You should be able to read books, presentations, papers, vulnerability disclosures, hacker zines. \\
Example Smashing The Stack For Fun And Profit, Aleph One
\item You should be able to find and use tools and frameworks\\
Example MITRE ATT\&CK, OWASP guides,
\end{list2}
\item Some are classic texts or from organisations and people you should KNOW after this course
\item A lot of resources are also linked throughout the course presentations
\end{list1}

\slide{Overview Diploma in IT-security}

\hlkimage{17cm}{kea-diplom-oversigt.png}


\slide{Go through exam Curriculum}

Primary literature:
\begin{list2}
\item \emph{The Art of Software Security Testing Identifying Software Security Flaws}
Chris Wysopal\\
ISBN: 9780321304865, AoST or the Green Book
\end{list2}


Our reading list is at:\\
{\small\url{https://zencurity.gitbook.io/kea-it-sikkerhed/softwaresikkerhed/lektionsplan}}
Required reading are:
\begin{list2}
    \item Basically the chapters from the books
\item Curriculum: AoST chapters 1-12
\item Curriculum: WAS chapter 1-21
\item Curriculum: Hacking chapter 1-3
\item We will now go through the curriculum and comment, ask questions
\end{list2}


\slide{Course Description}

From: STUDIEORDNING Diplomuddannelse i it-sikkerhed August 2018\\
Indhold
Modulet fokuserer på sikkerhedsperspektivet i software, blandt andet
programkvalitet og fejlhåndterings samt datahåndterings betydning for en
software arkitekturs sårbarheder.
Elementet introducerer også til forskellige designprincipper, herunder ”security by design”.

Viden  Den studerende har viden om:\\
Hvilken betydning programkvalitet har for it-sikkerhed ift.:
\begin{list2}
\item Trusler mod software
\item Kriterier for programkvalitet
\item Fejlhåndtering i programmer
\item Forståelse for security design principles, herunder:
\item Security by design
\item Privacy by design
\end{list2}

Færdigheder Den studerende kan:\\
Tage højde for sikkerhedsaspekter ved at:
\begin{list2}
\item Programmere håndtering af forventede og uventede fejl
\item Definere lovlige og ikke-lovlige input data, bl.a. til test
\item Bruge et API og/eller standard biblioteker
\item Opdage og forhindre sårbarheder i programkoder
\item Sikkerhedsvurdere et givet software arkitektur
\end{list2}

Kompetencer Den studerende kan:
\begin{list2}
\item Håndtere risikovurdering af programkode for sårbarheder.
\item Håndtere udvalgte krypteringstiltag
\end{list2}

Final word is the Studieordning which can be downloaded from\\
{\footnotesize \link{https://kompetence.kea.dk/uddannelser/it-digitalt/diplom-i-it-sikkerhed}\\
\link{Studieordning_for_Diplomuddannelsen_i_IT-sikkerhed_Aug_2018.pdf}}


\slide{Subject list walkthrough}

\begin{list2}
\item 1.Trusler mod software, oversigt over hvordan sårbarheder i software opstår
\item 2.Sikkerhedi udviklingsprocesser, Secure Software Development Lifecycle
\item 3.Sikkerhed i web applikationer
\item 4.Fuzzing af applikationer
\item 5.Softwareproblemer med håndtering af hukommelse
\item 6.Forbedret sikkerhed med opbygning af software i komponenter
\item 7.Håndtering af tekststrenge i software, herunder tegnsæt
\item 8.Netværksangreb mod software
\item 9.Audit af software, samt almindelige fejl der skal håndteres
\item 10.Security design og principper for sikkert design
\end{list2}

\slide{Deliverables and Exam}

\begin{list2}
\item Exam
\item Individual: Oral based on curriculum
\item Graded (7 scale)
\item Draw a question with no preparation. Question covers a topic
\item Try to discuss the topic, and use practical examples
\item Exam is 30 minutes in total, including pulling the question and grading
\item Count on being able to present talk for about 10 minutes
\item Prepare material (keywords, examples, exercises) for different topics so that you can use it to help you at the exam

\vskip 5mm
\item Deliverables:
\item 2 Mandatory assignments
\item Both mandatory assignments are required in order to be entitled to the exam.
\end{list2}


\slide{Mundtlig eksamen og formalia}

Eksamen varer samlet set i 30 minutter og forløber i 4 faser:

\begin{enumerate}
  \item Du trækker indledningsvist ét af de 10 ovenstående emner
  \item Du forklarer indledendeemnet støttet af egne slides i op til 10 minutter
  \item Herefter uddyber og diskuteres emnet i en dialog på 10 -- 15 minutter
  \item Afslutningsvist er der 5 minutters votering og karaktergivning
\end{enumerate}

Karakteren vil være en helhedsbedømmelseaf din viden om emnet samt din evne til at uddybe og diskutere relevante IT-sikkerhedsmæssige elementer. Der gives karakter efter 7 trins skalaen.


\end{document}
